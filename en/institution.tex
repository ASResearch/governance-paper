\section{Organization and Supervision: Nebulas Community Group}

In order to achieve the goal of ecological development as well as asset management and to support Nebulas’ goal of creating the Autonomous Metanet, the founding team will form the "Nebulas Community Group" together with the community. During the formation process, each organization's source of legitimacy, power, and boundaries will be strictly stipulated and constrained by one another. The three major organizations that comprise the Nebulas Community Group are:

\begin{enumerate}
	\item \textbf{Nebulas Council}: Oversees the legitimacy of the Nebulas governance process and the use of public assets within the Nebulas community; providing scaling advantages for the ecological development of Nebulas.
	\item \textbf{Nebulas Foundation}: Manage the Nebulas foundation's public assets, pool available resources and use the capital to offer efficiency advantages to the Nebulas ecosystem.
	\item \textbf{Nebulas Technical Committee}: Entrusted by the Nebulas Council; responsible for the productivity and quality verification of development projects as well as providing technical guidance and support to the community.
\end{enumerate}

\vspace{2em}

To ensure the independence of the Nebulas Community Group and to maintain checks and balances between them, there are two fundamental requirements:

\begin{enumerate}
	\item \textbf{The restraint of personal power}: All organizations are open to participants within the Nebulas community; however, a community member cannot hold a position in more than two organizations at the same time.
	\item \textbf{The restraint of organization power}: No single organization has the power to make independent decisions and to use public assets without the oversight of the other organizations.
\end{enumerate}

If it's necessary to introduce new principles, the three organizations must always be guaranteed independent operation and to be constrained by one another.

\subsection{Nebulas Council}

The Nebulas Council oversees the Nebulas governance process and the use of public assets of the Nebulas community to provide scale advantages for further ecological development of Nebulas.

\subsubsection{Directors}

The first directors of the Nebulas Council will comprise of 7 seats; of which, 3 seats will be nominated by the Nebulas Foundation and 4 seats will be elected via on-chain public voting within the community.

The numbers of nominations from the Nebulas Foundation are reduced by at least one seat every two years. After 6 years, the Nebulas Foundation can no longer nominate seats.

\subsubsection{Power}

\begin{enumerate}
	\item The Nebulas Council has the power to submit a proposal for a "\textbf{second vote}" (Reference: \ref{second-vote}).
	\item Appoint organizations such as the Nebulas Technical Committee or individuals to handle public affairs for the Nebulas community.
\end{enumerate}

\subsubsection{Obligations}

\begin{enumerate}
	\item Supervise the governance process.
	\item Supervise the safety of public assets such as community reserve funds.
\end{enumerate}

The Nebulas Council should ensure that the governance processes and the use of community public property are open and transparent. These methods include but are not limited to:

\begin{enumerate}
	\item Regularly updated asset use and community development via quarterly reports and other disclosure materials to the communities.
	\item Any technical upgrade, project application rejection, re-voting, etc... should be announced in a timely manner.
	\item All personnel elections and appointments should be announced on time.
\end{enumerate}

\subsubsection{Term of office}

The directors of the Nebulas Council have a term of two years and can be re-elected once their term is over.

\vspace{2em}


Community members have full oversight of the Nebulas Council. The directors of the Nebulas Council must comprise a report of their duties for their full term. The community will conduct a mid-term vote based on the submitted report to determine whether each Nebulas Council director will continue to serve.

If the director of the Nebulas Council fails to pass the midterm vote, the Nebulas Technical Committee will organize and supervise the election of a replacement Nebulas Council director. The directors who passed the mid-term review will temporarily complete the daily affairs of any directors who were removed until the election of the new Nebulas Council director is completed.

\subsubsection{Election method}

Except for the directors of the Nebulas Council nominated by the Nebulas Foundation, the directors of the Nebulas Council are elected through public on-chain voting. All members of the community who control at least one address on the Nebulas mainnet have the right to vote and to run for a seat on the council.

The first Nebulas Council election program has been proposed and will be supervised by the Nebulas Foundation. Future changes and iterations of the process must be performed through public, on-chain voting.

\subsubsection{Income}

\textbf{Total revenue}

10,000 NAS distributed over the 2-year term as described below.

\vspace{2em}

\textbf{Income distribution}

The revenue will be issued once every six months (four times in two years). The amount of each distribution every 6 months in order is: 1,500 NAS, 2,000 NAS, 3,000 NAS, 3,500 NAS (totaling 10,000 NAS). If the mid-term vote is not passed, the latter two payment will not be released.

\vspace{2em}

\textbf{Financial requirements}

To ensure the best interests of the economy and the continuity of the Nebulas Council, directors of the Nebulas Council must deposit 100,000 NAS into collateral for the initial 6 months of their term.

\vspace{2em}

\subsection{Nebulas Foundation}

The Nebulas founding team was formed in June of 2017; later, the Nebulas Foundation was established to take charge of the Nebulas team, their financial options, to ensure the normal operation of the project and to realize the development roadmap as promised in the \textit{Nebulas Non-technical Whitepaper}.

After all the technical points in the Nebulas Non-technicaal Whitepaper are fully completed, the Nebulas Foundation will manage the Foundation’s assets, pool resources, and use the capital to provide efficiency advantages for the ecological development of Nebulas.

\subsubsection{Members composition}

The managing directors of the Nebulas Foundation have no less than 5 seats including one chairman of the Nebulas foundation and one chief secretary.

\subsubsection{Power}

\begin{enumerate}
	\item Participate in the election of the chairman as well as the right to be elected.
	\item Participate in decision making in items such as foundation development and investment.
\end{enumerate}

\subsubsection{Obligations}

\begin{enumerate}
	\item Manage the assets and pool resources of the Nebulas Foundation.
	\item According to the needs of Nebulas, ensure the research and development of Nebulas and complete the technical features as promised in the Nebulas non-technical Whitepaper on time.
	\item Once a year, Nebulas Foundation directors will report to the Nebulas Council and continue to serve the Nebulas ecosystem.
\end{enumerate}

\subsubsection{Term in office}

The members of the Nebulas Foundation are appointed for a one year term. Afterwards, they are eligible for re-election.

\subsubsection{Inclusion method}

\textbf{Inclusion method}

The Nebulas Foundation adopts a capital-based entry system and all who receive the financial option reward to a certain amount are automatically eligible to become a Foundation Managing Director. Subsequently, all eligible members have the option to waive becoming a Foundation Managing Director. If there are less than 5 members within the Board of Managing Directors, they will be ranked according to their financial option reward size.

\vspace{2em}

\textbf{Chairman of the Nebulas Foundation}

The chairman of the Nebulas Foundation is elected among the current members of the Nebulas Foundation. Each member of the Nebulas Foundation has the right to vote and to be elected.

To become the chairman of the Nebulas Foundation, a member must receive a minimum of 50\% of all casted votes. During voting and none of the directors receive 50\% of the votes, those with either no votes or the minimum amount of votes are eliminated and voting will be restarted until a member receives an approval rating of 50\% or greater.

\vspace{2em}

\textbf{Chief Secretary of the Nebulas Foundation}

The Chief Secretary of the Nebulas Foundation is appointed by the Chairman among current members of the Nebulas Foundation.

\vspace{2em}

\textbf{Managing Director of the Nebulas Foundation}

The Managing Director of the Nebulas Foundation is appointed by the Chairman among current members of the Nebulas Foundation.

\vspace{2em}

\textbf{Recall}

The Nebulas Foundation can remove any member of the Nebulas Foundation through internal resolutions and the results must be disclosed to the community.

The removed member(s) of the Nebulas Foundation has the right to address to the community publicly and call for a public, on-chain vote to request a re-vote for reinstatement to the Foundation.

\subsubsection{Income}

\textbf{Total revenue}

\begin{enumerate}
	\item Nebulas Foundation Salary and relevant option reward.
    \item Enjoy the benefits of investing within the Nebulas Foundation’s eco-investment, etc.
\end{enumerate}

\vspace{2em}

\textbf{Financial requirements}

To ensure the best interests of the economy and the continuity of the Nebulas Foundation, official entry requires 50,000 NAS collateral deposit which is unlocked after 6 months.

\subsection{Nebulas Technical Committee}

The Nebulas Technical Committee was established September, 2018. Since its establishment, the Nebulas Technical Committee has adhered to the spirit of openness, sharing and transparency. The Nebulas Technical Committee is committed to promoting the decentralization and community collaboration of research and development of Nebulas technology. 

Since the establishment of the Nebulas Council, the Nebulas Technical Committee, composed initially of core members of the Nebulas team will complete its historical mission and will transform into a community-based organization. Entrusted by the Nebulas Council, the Nebulas Technical Committee is responsible for the productivity and quality verification of the Nebulas project, providing technical guidance and support to the community.

\subsubsection{Members composition}

The number of Nebulas Technical Committee members is not limited.

\subsubsection{Power}

\begin{enumerate}
	\item The power to initiate and review of community proposals.
	\item Enjoy the honor of being included with a team of experts pertaining to Nebulas technology.
\end{enumerate}

\textbf{Obligations}

\begin{enumerate}
	\item Quality supervision of community proposals.
	\item Issue relevant test and technical rating reports.
\end{enumerate}

\subsubsection{Term in office}

Members of the Nebulas Technical Committee will serve a one year term and may be re-elected afterwards.

\subsubsection{Inclusion method}

The Nebulas Technical Committee adopts a combination of self-recommendation and community recommendation which is publicly reported to the community. The appointment will be organized by the Nebulas Council.

\subsubsection{Income}

\textbf{Total revenue}

\begin{itemize}
	\item Commission (issued monthly).
	\item Consulting fee for project review and supervision.
\end{itemize}

\vspace{2em}

\textbf{Financial requirements}

To ensure the consist interests of the economy and the continuity of the Nebulas Technical Committee policy, members of the committee require 25,000 NAS in collateral when they formally join the committee. The collateral is returned 3 months after dismissal from the Nebulas Technical Committee.
