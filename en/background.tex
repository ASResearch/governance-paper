\section{Background}

The goal of Nebulas governance is to achieve the Nebulas vision and its focuses on decentralized collaboration. Before introducing the details of Nebulas governance, first you need to know the difficulty of collaboration via blockchain and the issues that Nebulas wants to solve.

\label{background}

Human beings are social and we are not strangers to Collaboration. Even Robinson on the island had a set of models to collaborate with Friday~\cite{robinson}. Collaboration itself does not have any absolute advantages or disadvantages. In different situations, one or more collaborations methods may be suitable for the specific situation. With the development of science and technology, the current situation has gone from face-to-face cooperation to global collaboration across locations and organizations. The goal of the collaboration has also become more variable with requirements changing from physical to virtual and periods becomes longer and more flexible.

Nebulas does not want to subvert other forms of collaboration nor exclude multiple ways of collaborating. Nebulas is trying to find a more appropriate way to collaborate and complement others in new situations. The new system of collaboration has the following features:

\begin{itemize}
	\item \textbf{Information interaction is changing from simple to complex.}

	The birth of the decentralized, digital cryptocurrency Bitcoin allows the blockchain to record transaction information. The second-generation blockchain system, such as Ethereum proposed a smart contract with Turing completeness which also allowed blockchain becomes programmable. From there, more and more data and asset interaction result in new problems and unique situations such as on-chain, off-chain, and cross-chain.

	\item \textbf{Users roles are increasing.}

	In the early Bitcoin community, there were only miners and token-holders. Ethereum also allowed new groups such as developers and users.  As more people learn about blockchain, it becomes a challenge to distribute rights and responsibilities to different users based on their roles.

\end{itemize}

Problems such as:

\begin{enumerate}
	\item 

	\textbf{Centralized governance cannot cope with new and complex situations.}

	Blockchain is essentially a decentralized, trustless, game-based autonomous system. Its true charm is the open collaboration model based on the consensus mechanism and decentralization ideology~\cite{whitepaper}. However, some blockchain projects simply do the opposite and use centralization as way to govern; such as directly arbitrating hackers through the core arbitration court. The legitimacy and fairness of this approach is difficult to guarantee. Faced with complex data interaction patterns and rich user roles, the centralization of a single evaluation criteria is difficult to achieve which causes community members to rebel. On January 11, 2019, the EOS Authority initiated a vote on whether to delete the ECAF (The EOSIO Core Arbitration Forum), and the percentage of support was as high as 98\%~\cite{DeleteECAF}。

	\item 

	\textbf{Existing decentralized governance rules are not uniform.}

	In the Bitcoin community users that with different roles such as miners and holders have differing rules and it's not clear who should follow what rule. Such decentralized governance methods are likely to cause unclear community development goals which makes it difficult to effectively organize and execute a upgrade.

	\item 

	\textbf{Traditional decentralized collaboration is commonly a tragedy~\cite{TragedyOfTheCommons}.}

	Traditional decentralized collaborative projects, such as a large number of open source communities, have unclear investment models and their funding source is often based on donation. Upgrades and improvements rely too much on the developers' interests. There are frequent problems with slow evolution of the ecosystem due to differing opinions. Today, there are more people using public resources (such as open source code) and fewer people contributing. Open source projects that rely on large companies and enterprises for donations are often blocked by the development direction of the large donors and become affiliates of enterprises.

	Tokens that exist within a blockchain ecosystem gives us the opportunity to solve the basic dilemma of decentralized collaboration by providing sustainable incentives to build a prosperous economy.

	\item

	\textbf{Early blockchain project consensus mechanism incentives are not comprehensive and community participation is low.}

	The Proof of Work (PoW~\cite{pow}) system used by Bitcoin only focuses on incentives for the miners; this single incentive cannot cope with the enrichment of all users regardless of their role. Ethereum with its decentralized nature has been repeatedly criticized for its slow upgrade process due to proposed improvements requiring recognition by the majority of the community and execution by the miners. This shows how difficult it is to unify the opinions of an entire ecosystem. The phenomenon of “doing nothing” has led to a low participation rate with Ethereum upgrade proposals resulting in delayed implementation and damage to its ecological development.

\end{enumerate}

There is currently no perfect solution to solve the above problems. We realize that in the new world with unprecedented complexity, the birth of new technologies is expected.
