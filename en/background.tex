\section{Background}

The goal of Nebulas governance is to achieve the vision of Nebulas and its focuses of creating decentralized collaboration. Before introducing the details of Nebulas governance, first you need to understand the difficulty of collaboration via blockchain and the issues that Nebulas wants to solve.

\label{background}

Humans are social creatures and are not strangers to \textbf{Collaboration}. Even Robinson Crusoe on the island had a set of entities to collaborate with including Friday~\cite{robinson}. Collaboration itself does not have any absolute advantages or disadvantages and even in different situations, one or more collaboration methods may be suitable. With the development of science and technology, current collaboration methods has gone from face-to-face cooperation to a global collaboration network across locations and within organizations. The goal of collaboration has also become more unique with requirements changing from physical to virtual and operation periods become more flexible.

Nebulas does not want to subvert other forms of collaboration nor exclude multiple ways of collaborating. Instead, Nebulas is trying to find a more appropriate way to collaborate and to complement others in new situations. The new collaboration structure has the following features:

\begin{itemize}
	\item \textbf{Information interaction is changing from simple to complex.}

	The birth of the decentralized, digital cryptocurrency, Bitcoin allows blockchain to record transaction information. The second-generation blockchain system, such as Ethereum introduced smart contracts with Turing completeness which allowed blockchain to become programmable. From there, more and more data and asset interaction result in new problems and unique situations such as handling on-chain, off-chain, and cross-chain interaction.

	\item \textbf{Users roles are increasing.}

	In the early Bitcoin community, there were only miners and token-holders. Ethereum added new groups to blockchain ecosystems such as developers and decentralized application users. As more people learn about blockchain, it becomes a challenge to distribute power and responsibilities to different users based on their unique roles.

\end{itemize}

Some of the current problems are:

\begin{enumerate}
	\item 

	\textbf{Centralized governance cannot cope with new and complex situations.}

	Blockchain is essentially a decentralized, trust-less, game-based autonomous system. It's true appeal is the open collaboration model based on consensus mechanisms and decentralization ideology~\cite{whitepaper}. However, some blockchain projects simply do the opposite and use centralization as way to govern; such as directly arbitrating hackers through the core arbitration court. The legitimacy and fairness of this approach is difficult to guarantee. Faced with complex data interaction patterns and expanded user roles, the centralization of a single evaluation criteria is difficult to achieve which causes community members to rebel. For example, on January 11, 2019, the EOS Authority initiated a vote on whether to terminate the ECAF (The EOSIO Core Arbitration Forum); the percentage of support was as high as 98\%~\cite{DeleteECAF}.

	\item 

	\textbf{Existing decentralized governance rules are not uniform.}

	In the Bitcoin community users with different roles such as miners and holders have differing rules and it's not clear who should follow what rule. Such decentralized governance methods are likely to cause unclear community development goals which makes it difficult to effectively organize and execute a upgrade.

	\item 

	\textbf{Traditional decentralized collaboration is commonly a tragedy~\cite{TragedyOfTheCommons}.}

	Traditional decentralized collaborative projects, such as a large number of open source communities, have unclear investment models and their funding source is often based on donation. Upgrades and improvements rely too much on the developers interests and there are frequent problems with slow evolution of the ecosystem due to differing opinions. Today, there are more people using public resources (such as open source code) but fewer people contributing. Many open source projects now rely on large companies and enterprises for donations and often motivated to continue development in the direction warranted by the large donors rather than their opinions - in essence, they become part of the corporation.

	Tokens that exist within a blockchain ecosystem give us the opportunity to solve the basic dilemma of decentralized collaboration by providing sustainable incentives to build a prosperous economy free from corporate control.

	\item

	\textbf{Early blockchain project consensus mechanism incentives are not comprehensive and community participation is low.}

	The Proof of Work (PoW~\cite{pow}) system used by Bitcoin only focuses on incentives for the miners; this single incentive system cannot cope with the enrichment of all users regardless of their role. Ethereum with its decentralized nature has been repeatedly criticized for its slow upgrade process due to all proposed improvements requiring approval by the majority of the community and execution by node operators. This shows how difficult it is to unify the opinions of an entire ecosystem. The phenomenon of "doing nothing" has led to a low participation rate with Ethereum upgrade proposals resulting in delayed implementation and damage to its ecological development.

\end{enumerate}

There is currently no perfect solution to solve the above problems and we realize that in the new world with unprecedented complexity, the creation of new technology is expected.
