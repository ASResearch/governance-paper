\section{Overview}

Nebulas Governance\textbf{ takes on-chain governance as the core, the on-chain assets as the main object, and on-chain interaction as the basic collaboration way} to solve the governaance ddifference in the above situations, and design a better decentralized collaboration way to achieve the Nebulas vision.

\subsection{Technology}

The open source public chain Nebulas was designeed to achieve the vision in 2017, and it provides the technical support for Nebulas governance.

Nebulas is autonomous metanet. Nebulas uses hyper-mapped metadata to solve complex data and interaction problems, and it has the core ability to measure the vaalue of on-chain data (Nebulas Rank, NR); uses new consensus mechanism and upgrade capabilities to solve complex collaboration problems and provides lasting positive incentive (Nebulas Incentive, NI), even upgrading without hard-fork (Nebulas Force, NF). Nebulas tries to use techonology to reduce the friction and governance costs that are often present in human governance and change collaborative relationships, and promote healthy development of communities. For details of the Nebulas technology, please refer to the Nebulas Technology Whitepaper~\cite{TechWhitepaper}.

\subsection{Three basic rights}
\label{rights}

All complex systems begin with the development of basic rules that follow logical steps. Since the basic component of blockchain assets is an “address” this is also the basic unit of the Nebulas community governance. Therefore, we formally propose three basic rights for each Nebulas address:

\begin{enumerate}
	\item The right to own and utilize assets on Nebulas,
	\item The right to initiate a proposal,
	\item The right to vote.
\end{enumerate}

Nebulas strongly believes that each address has the above listed fundamental rights within the system and will not be infringed upon under any circumstances. Any user who has access to a Nebulas address via their unique private key has a “right” to control their assets. Without any absolute centralized organization or individuals, each and every member of the Nebulas community has the freedom to use the mainnet and participate in the decision making process. Members can also participate in the production and construction of community approved projects. 

Nebulas governance is based on these three rights. Simply put, anyone can create a proposal, share it with the community, and ultimately, have the community approve their proposal via an on-chain voting system. This means that the future of Nebulas is in the hands of every participating community member!

\subsection{Governance range}

The public assets primarily controlled by Nebulas governance include:

\begin{enumerate}
	\item Intellectual property, including public, open source code (such as Nebulas Mainnet upgrades and other related codes that affect the public interest of Nebulas).
	\item NAS funds Reserved for the Community Ecosystem according to the Nebulas Whitepaper.
\end{enumerate}

In general, blockchain is a network that tracks collaboration relationships, as well as a network that tracks “assets” of incentive cooperation. In a system that is absent of any centralized power, public assets should be managed by all community members.

But at the same time, the scope of Nebulas governance is limited to Nebulas public assets, not all assets on the Nebulas. Nebulas governance provides basic governance tools for the Nebulas community. Organizations in the Nebulas community (such as DApp project parties, exchanges, etc.) can use the Nebulas governance tools (such as NAT on-chain voting) to promote the ecological development of their projects, but the Nebulas Community Group Will not "take up" the role of judge. Besides, the off-chain events of the Nebulas community members should comply with local laws and regulations. In the previous chapter ~\ref{background} has been described, different situations should adopt matching governance, and Nebulas governance will not violate the original intention of the design, blindly expand the governance range.

\subsection{Features}

There are three features of Nebulas governance:

\begin{enumerate}
	\item 

	\textbf{Rule hologram, oopen and transparent.} 

	Everyone survives and develops under normal rules. At the same time, the rules of iterative rules are defined by normal rules.


	\item 

	\textbf{The decentralized collaboration of a benefit economy. }

	\begin{enumerate}
		\item 

		\textbf{Decentralizate the process of community collaboration:} The on-chain governance is the core of Nebulas governance and assists community members in monitoring.
	
		\item 

		\textbf{Decentralizate the governance of public assets:} As a decentralized community with asset attributes:

		\begin{itemize}
			\item Nebulas Community Group protects the process legal and powers are mutually restricted. No organization or individual has supreme power, and no organization or individual can directly use public assets.
			\item provides technical support for asset governance and security through the original Proof of Devotion (PoD).
		\end{itemize}

	\end{enumerate}

	\item 

	\textbf{Complete incentive, high community participation.}
	
	Lasting positive incentives are the core of community organizations and the cornerstone of autonomy.

	The Nebulas original Core Nebulas Rank can be combined with a variety of parameters to determine the contribution of an address on the entire economic system ~\cite{yellowpaper}. Based on this, not only miners and currency users, but also developers, active users, and other different roles can be a relatively regular quantitative contribution to the entire ecology, and can also be compared with each other, so that the Nebulas can inspire everyone in the ecosystem according to their contributions.

	Moreover, Nebulas “assetization” Nebulas Rank by increasing the contribution of the economy and actively participating on-chain governance (such as on-chain voting), Nebulas users can get the  NAT rewards by contributing to the community and get positive incentives. This incentive is the native incentives of the algorithm, implemented through technical support rather than single command, using tools to reduce the likelihood of individual manipulation of the entire network.

	There are three basic pieces of knowledge about motivation:

	\begin{enumerate}
		\item Positive incentives are the basis for ensuring good benefits for everyone. Incorrect incentives can lead to bad money driving out good money.
		\item incentives should be continuously, and short-sighted incentives can also cause irreversible results.
		\item the scale of incentive should be appropriate.
	\end{enumerate}

	Nebulas always regards incentives as an essential part in designing technical features and Nebulas economy. Positive incentives are expected to benefit community members more equitably and significantly increase community engagement.
	
	\item 

	\textbf{Inclusive and efficient collaboration. }
	
	Based on the Nebulas autonomous metanet, there is no need for hard forks to self-evolve. In the Nebulas community, once a proposal is approved through the Nebulas on-chain voting, it can be upgraded immediately and iterated quickly. If you face problems, you can also try again soon, embrace complexity, and be fault-tolerant. It will not be like the early blockchain system, such as the public chain, Ethereum, bound by consensus mechanisms or immature technologies and strategies.

	While technically efficient, Nebulas governance also has a transparent and straightforward process (see \ref{governance}) to improve collaboration efficiency.

\end{enumerate}
