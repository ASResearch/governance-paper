\section{Overview}

Nebulas governance \textbf{takes chain management as the core, chain assets as the main governance object, and chain interaction as the basic coordination method} to solve the governance dilemma under the above new scenarios and designs a better decentralized collaboration model to realize the Nebulas vision.

\subsection{Technology}

The open source, public blockchain Nebulas was designed to achieve this vision in 2017 and has provided the technical framework for Nebulas governance.

Nebulas is an autonomous metanet which uses hyper-mapped metadata to solve complex data and interaction problems. It has the core ability to measure the value of on-chain data (via Nebulas Rank, NR); uses new consensus mechanism and upgrade capability to solve complex collaboration problems; provides lasting positive incentive (Nebulas Incentive, NI) for its users; has the ability to upgrade without a hard-fork (Nebulas Force, NF). Nebulas tries to use technology to reduce the friction and governance costs that are often present in human governance, to change collaborative relationships and to promote healthy development of communities. For details of Nebulas' technology, please refer to the Nebulas Technical Whitepaper~\cite{TechWhitepaper}.


\subsection{Three basic rights}
\label{rights}

All complex systems begin with the development of basic rules that follow logical steps. Since the basic component of blockchain assets is an “address” this is also the basic unit of the Nebulas community governance. Therefore, we formally propose three basic rights for each Nebulas address:

\begin{enumerate}
	\item The right to own and utilize assets on Nebulas,
	\item The right to initiate a proposal,
	\item The right to vote.
\end{enumerate}

Nebulas strongly believes that each address has the above listed fundamental rights within the system and will not be infringed upon under any circumstances. Any user who has access to a Nebulas address via their unique private key has a “right” to control their assets. Without any absolute centralized organization or individuals, each and every member of the Nebulas community has the freedom to use the mainnet and participate in the decision making process. Members can also participate in the production and construction of community approved projects. 

Nebulas governance is based on these three rights. Simply put, anyone can create a proposal, share it with the community, and ultimately, have the community approve their proposal via an on-chain voting system. This means that the future of Nebulas is in the hands of every participating community member!

\subsection{Governance scope}


The public assets primarily controlled by Nebulas governance include:

\begin{enumerate}
	\item Intellectual property, including public, open source code (such as Nebulas Mainnet upgrades and other related codes that affect the public interest of Nebulas).
	\item NAS funds Reserved for the Community Ecosystem according to the Nebulas Whitepaper.
\end{enumerate}

In general, blockchain is a network that tracks collaboration relationships, as well as a network that tracks “assets” of incentive cooperation. In a system that is absent of any centralized power, public assets should be managed by all community members.

At the same time, the scope of Nebulas governance is limited to Nebulas public assets and not all assets on the Nebulas. Nebulas governance provides basic governance tools for the Nebulas community. Organizations in the Nebulas community (such as DApp project parties, exchanges, etc.) can use the Nebulas governance tools (such as NAT on-chain voting) to promote the ecological development of their projects; however, the Nebulas Community Group Will not "take up" the role of judge. With off-chain events, Nebulas community members should comply with local laws and regulations. In the previous chapter ~\ref{background} described different situations that should adopt a matching governance model; Nebulas governance will not violate the original intention of the design to blindly expand the governance range.

\subsection{Features}

There are three primary features of Nebulas governance:

\begin{enumerate}
	\item 

	\textbf{Rule "hologram," open and transparent.} 

	Everyone co-exists and develops under standardized rules. At the same time, any new requirements are defined by the base rules.

	\item 

\textbf{The decentralized collaboration of a prospering economy.}


	\begin{enumerate}
		\item 

		\textbf{Decentralize the process of community collaboration:} On-chain governance is the core of Nebulas governance and allows for community oversight of the process.
		
		\item 

		\textbf{Decentralize the governance of public assets:} As a decentralized community with asset attributes:

		\begin{itemize}
			\item The Nebulas Community Groups will ensure the legitimacy of the governance process and the mutual restriction of power; no organization or individual has absolute power and no organization or individual can directly use public assets.
			\item Provides technical support for asset governance and security through the original Proof of Devotion (PoD) consensus mechanism .
		\end{itemize}

	\end{enumerate}

	\item 

	\textbf{Incentivize the community for a high participation rates}
	
	Lasting positive incentives are the core of community organizations and the cornerstone of autonomy.

	The Core Nebulas Rank can be combined with a variety of parameters to determine the contribution of an address within the entire economic system ~\cite{yellowpaper}. Based on this, not only miners and currency users, but also developers, active users and others in different roles can be a relatively regular quantitative contribution to the entire ecosystem. All users can also be compared with one another and in return, Nebulas can inspire everyone in the ecosystem according to their contributions.

	Moreover, by utilizing the "asset-based" Nebulas Rank and actively participating on-chain governance (such as on-chain voting), users can receive NAT incentives by contributing to the community and ecosystem. The NAT token is the native incentive of the Nebulas Rank algorithm which is implemented through technical capabilities rather than personal intervention which reduce the likelihood of individual manipulation on the network.

	Three basic perceptions about ecosystem motivation:

	\begin{enumerate}
		\item Positive incentives are the basis for ensuring good benefits for everyone. Incorrect or uneven distribution of incentives can lead to bad money driving out good money.
		\item Incentives should be continuous; short-lived incentives can cause irreversible, negative results.
		\item the scale of incentives should be appropriate.
	\end{enumerate}

	Nebulas always regards incentives as an essential part in designing technical features of the Nebulas economy. Positive incentives are expected to benefit community members more equitably and significantly increase community engagement.
	
	\item 

	\textbf{Inclusive and efficient collaboration.}
	
	Since Nebulas is a true autonomous metanet, there is no need for hard forks to self-evolve. Within the Nebulas community, once a proposal is approved via on-chain voting, an upgrade can be completed and iterated immediately. If a problem arises on the network, improvements can be quickly be released to the entire network. Future problems on Nebulas will not be like those on existing public chains such as Ethereum which are bound by is consensus mechanisms or immature technologies and strategies.

	While technically efficient, Nebulas governance also has a transparent and straightforward process (see \ref{governance}) to improve collaboration efficiency.

\end{enumerate}
