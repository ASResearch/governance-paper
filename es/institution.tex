\section{Organización y supervisión: Grupo Comunitario de Nebulas}

Con el fin de alcanzar el objetivo del desarrollo ecológico, así como la gestión de activos y para apoyar el objetivo de Nebulas de crear la \textit{metanet autónoma}, el equipo fundador formará el \textbf{Grupo Comunitario de Nebulas} con ayuda de la comunidad. Durante el proceso de formación, la fuente de legitimidad, el poder y los límites de cada organización estarán estrictamente estipulados y limitados entre sí. Las tres organizaciones principales que componen el Grupo Comunitario de Nebulas son:

\begin{enumerate}
	\item \textbf{Concejo de Nebulas}: supervisar la legitimidad del proceso de gobernanza y el uso de los activos públicos dentro de la comunidad, proporcionando ventajas de escalabilidad para el desarrollo ecológico de Nebulas.

	\item \textbf{Fundación Nebulas}: gestionar los activos públicos de la Fundación Nebulas, agrupar los recursos disponibles y utilizar el capital para ofrecer ventajas de eficiencia al ecosistema.

	\item \textbf{Comité Técnico de Nebulas}: acreditado por el Consejo de Nebulas, es responsable de la productividad y la verificación de la calidad de los proyectos en desarrollo, como así también de proporcionar orientación técnica y apoyo a la comunidad.

\end{enumerate}

\vspace{2em}

Para asegurar la independencia del Grupo Comunitario de Nebulas y para mantener los controles y equilibrios entre sus órganos, se han dispuesto dos requisitos fundamentales:

\begin{enumerate}
	\item \textbf{Restricción del poder personal}: todas las organizaciones están abiertas a los participantes dentro de la comunidad de Nebulas; sin embargo, un miembro de la comunidad no puede tener una posición en más de dos organizaciones al mismo tiempo.
	\item \textbf{Restricción del poder de las organizaciones}: ninguna organización tiene el poder de tomar decisiones de forma unilateral, ni de utilizar los bienes públicos sin la supervisión de las demás organizaciones.
\end{enumerate}

Si es necesario introducir nuevos principios, se debe garantizar a los tres órganos la libertad de funcionamiento y el control entre ellos.

\subsection{Concejo de Nebulas}

El Consejo de Nebulas supervisa el proceso de gobernanza y el uso de los activos públicos de la comunidad, proporcionando ventajas de escalabilidad para su desarrollo ecológico.

\subsubsection{Directores}

Los primeros directores del Consejo de Nebulas tendrán siete escaños, de los cuales tres serán nominados por la Fundación Nebulas y cuatro serán elegidos por votación pública dentro de la comunidad.

El número de nominaciones de la Fundación Nebulas se reduce en por lo menos un escaño cada dos años. Después de seis años, la Fundación Nebulas ya no podrá nominar puestos.

\subsubsection{Poderes}

\begin{enumerate}
	\item Remitir propuestas para un \textbf{segundo voto} (\ref{second-vote}).
	\item Designar organizaciones como el Comité Técnico de Nebulas o individuos para manejar los asuntos públicos de la comunidad de Nebulas.
\end{enumerate}

\subsubsection{Obligaciones}

\begin{enumerate}
	\item Supervisar el proceso de gobernanza.
	\item Supervisar la salvaguarda de activos públicos tales como los fondos de reserva comunitarios.
\end{enumerate}

El Concejo de Nebulas debe garantizar que el proceso de gobernanza y el uso de propiedad pública comunitaria se realice de una forma abierta y transparente. Estos métodos incluyen, pero no se limitan, a:

\begin{enumerate}
	\item Comunicar regularmente el uso de activos y el desarrollo de la comunidad mediante reportes trimestrales y otros materiales de divulgación a las comunidades.
	\item Toda actualización técnica, rechazo de aplicación de proyecto, reedición de una votación, etcétera, deberá ser anunciada en tiempo y forma.
	\item Todas las elecciones personales y los nombramientos deben anunciarse a tiempo.
\end{enumerate}

\subsubsection{Duración del mandato}

Los directores del Concejo de Nebulas tienen un mandato de dos años, finalizado el cual podrán ser reelegidos.

\vspace{2em}

Los miembros de la comunidad tendrán plena supervisión sobre el Concejo de Nebulas. Sus directores  deberán elaborar un informe, enumerando en él sus deberes durante todo su mandato; la comunidad llevará a cabo una votación a mitad de período, basada en ese informe, para determinar si cada director del Concejo continuará en servicio.

Si el director del Concejo no aprueba la votación de mitad de período, el Comité Técnico organizará y supervisará la elección de un director sustituto. Los directores que hayan aprobado la revisión de medio término deberán cubrir temporalmente las tareas corrientes de cualquier director que haya sido removido, hasta que se complete la elección de un nuevo director.

\subsubsection{Método de elección}

Con excepción de los directores nominados por la Fundación Nebulas, los directores del Concejo serán elegidos mediante votación pública \onchain. Todos los miembros de la comunidad que controlen al menos una dirección en la red principal de Nebulas tendrán derecho a votar y a postularse para un puesto en el concejo.

El primer programa electoral del Concejo será propuesto y supervisado por la Fundación Nebulas. Los futuros cambios e iteraciones del proceso deben realizarse a través de la votación pública \onchain.

\subsubsection{Ingresos}

\textbf{Total de ingresos}

Se distribuirán 10 000 NAS en un plazo de dos años, tal como se describe más abajo.

\vspace{2em}

\textbf{Distribución de los ingresos}

Los ingresos se emitirán una vez cada seis meses, del siguiente modo: 1500 NAS el primer semestre, 2000 NAS el segundo semestre, 3000 NAS el tercer semestre, 3500 NAS el último semestre. Si no se aprueba la votación de medio término, los dos últimos pagos no serán liberados.

\vspace{2em}

\textbf{Requerimientos financieros}

Para resguardar los intereses de la economía y la continuidad del Concejo de Nebulas, los directores deberán depositar 100 000 NAS como prenda de garantía durante los primeros seis meses de su mandato.

\vspace{2em}

\subsection{Fundación Nebulas}

El Equipo Fundador de Nebulas se formó en junio de 2017; poco después se estableció la Fundación Nebulas para hacerse cargo del equipo y de sus opciones financieras, con el fin de asegurar el funcionamiento normal del proyecto y para realizar el plan de desarrollo mencionado en el \ntechw.

Una vez realizados todos los puntos técnicos del \ntechw, la Fundación Nebulas podrá disponer de sus activos, reunir recursos y utilizar el capital para incrementar la eficiencia del desarrollo ecológico de Nebulas.

\subsubsection{Composición de los miembros}

Los directores generales de la Fundación Nebulas tendrán no menos de cinco escaños, incluyendo un Presidente y un Secretario en Jefe.

\subsubsection{Poderes}

\begin{enumerate}
	\item Participar en la elección del presidente, así como ejercer el derecho a ser elegidos.
	\item Participar en la toma de decisiones en temas como el desarrollo de fundaciones e inversiones.
\end{enumerate}

\subsubsection{Obligaciones}

\begin{enumerate}
	\item Gestionar los activos y recursos de la Fundación Nebulas.
	\item De acuerdo con las necesidades, asegurar en tiempo y forma la investigación y desarrollo de Nebulas, y completar sus características técnicas de acuerdo al \ntechw.
	\item Una vez al año, los directores de la Fundación elevarán un informe de gestión al Concejo.
\end{enumerate}

\subsubsection{Duración del mandato}

Los miembros de la Fundación Nebulas son nombrados por un período de un año. Cumplido el mandato pueden postularse nuevamente y ser reelegidos.

\subsubsection{Método de inclusión}

\textbf{Método de inclusión}

La Fundación Nebulas adopta un sistema de ingreso basado en capitales. Quienes hayan recibido un mínimo determinado de opciones financieras (a cambio de su contribución de capital) serán automáticamente elegibles para postularse como Directores Generales de la Fundación. De la misma manera, todos los miembros elegibles tendrán la opción de renunciar a esos cargos. De existir menos de 5 miembros en el Consejo de Administración, serán clasificados de acuerdo con el tamaño de la opción financiera en su haber.

\vspace{2em}

\textbf{Presidente de la Fundación Nebulas}

El presidente de la Fundación será elegido entre los miembros actuales de la misma. Cada miembro de la Fundación Nebulas tiene derecho a votar y a ser elegido.

Para consagrarse como presidente de la Fundación Nebulas el candidato debe recibir un mínimo del 50\% de todos los votos emitidos. Si en la votación ninguno de los candidatos recibe el 50\% de los votos, aquellos que no hayan recibido votos, o la cantidad mínima de votos establecida, serán eliminados y la votación se reiniciará hasta que un candidato reciba el 50\% o más de los votos.

\vspace{2em}

\textbf{Secretario en Jefe de la Fundación Nebulas}

La Secretaría Principal de la Fundación Nebulas será elegida por el Presidente de la Fundación Nebulas entre los miembros actuales de la misma.

\vspace{2em}

\textbf{Director administrativo de la Fundación Nebulas}

El Director Gerente de la Fundación Nebulas será nombrado por el Presidente entre los miembros actuales de la misma.

\vspace{2em}

\textbf{Retiros}

La Fundación Nebulas puede remover a cualquier miembro de la misma a través de resoluciones internas; los resultados deberán ser divulgados a la comunidad.

Los miembros removidos tienen el derecho de dirigirse a la comunidad públicamente y solicitar un voto público \onchain para exigir una nueva votación con el fin de reintegrarse a la Fundación.

\subsubsection{Ingresos}

\textbf{Ingresos totales}

\begin{enumerate}
	\item Salarios de la Fundación Nebulas y recompensas de opción relevante.
	\item Inversiones dentro del ecosistema de inversiones de la Fundación, etc.
\end{enumerate}

\vspace{2em}

\textbf{Requerimientos financieros}

Para asegurar los mejores intereses de la economía y la continuidad de la Fundación Nebulas, el ingreso a la misma requiere un depósito en garantía de 50 000 NAS que se desbloqueará luego de seis meses en ejercicio.

\subsection{Comité Técnico de Nebulas}

El Comité Técnico de Nebulas fue establecido en septiembre de 2018, adhiriendo desde su creación al espíritu de apertura, intercambio y transparencia. El Comité está comprometido a promover la descentralización y la colaboración comunitaria en la investigación y desarrollo de la tecnología de Nebulas.

Desde el establecimiento del Concejo, su Comité Técnico —compuesto inicialmente por miembros fundadores del equipo de Nebulas— completará su misión histórica y se transformará en una organización basada en la comunidad. Encomendado por el Concejo, el Comité Técnico de Nebulas es responsable de la productividad y la verificación de la calidad del proyecto de Nebulas, proporcionando orientación técnica y apoyo a la comunidad.

\subsubsection{Composición de los miembros}

No hay límites en el número de miembros del Comité Técnico.

\subsubsection{Poderes}

\begin{enumerate}
	\item El poder de iniciar y revisar las propuestas de la comunidad.
	\item Disfrutar del honor de estar incluido en un equipo de expertos en tecnología.
\end{enumerate}

\textbf{Obligaciones}

\begin{enumerate}
	\item Supervisión de la calidad de las propuestas de la comunidad.
	\item Emitir los informes de pruebas y calificaciones técnicas pertinentes.
\end{enumerate}

\subsubsection{Duración del mandato}

Los miembros del Comité Técnico de Nebulas servirán por un período de un año y podrán ser reelegidos posteriormente.

\subsubsection{Método de inclusión}

El Comité Técnico de Nebulas adopta una método mixto de auto-recomendación y recomendación comunitaria que es reportada públicamente a la comunidad. La designación será organizada por el Concejo de Nebulas.

\subsubsection{Ingresos}

\textbf{Ingresos totales}

\begin{itemize}
	\item Comisión (emitida mensualmente).
	\item Honorarios de consultoría para la revisión y supervisión de los proyectos.
\end{itemize}

\vspace{2em}

\textbf{Requerimientos financieros}

Para asegurar los intereses de la economía y la continuidad de la política del Comité Técnico de Nebulas, sus miembros deberán depositar 25 000 NAS en garantía al momento de unirse formalmente al comité. Ese depósito en garantía se devolverá 3 meses después de su salida del comité.