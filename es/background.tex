\section{Antecedentes}

El objetivo de la gobernanza de Nebulas es el de hacer realidad su visión de colaboración descentralizada. Antes de presentar los detalles de la gobernanza es necesario entender los problemas que plantea la colaboración descentralizada y la forma en que Nebulas los busca resolver.

\label{background}

Los seres humanos son criaturas sociales y no son ajenos a la \textit{colaboración}; incluso \textit{Robinson Crusoe}, en su isla, tenía un conjunto de entidades con las que colaborar, incluyendo a su amigo \textit{Viernes}.~\cite{robinson}

La colaboración en sí misma no tiene ventajas o desventajas absolutas, e incluso ante distintas situaciones uno o más métodos de colaboración pueden resultar adecuados.

Con el desarrollo de la ciencia y la tecnología, los métodos actuales de colaboración han pasado de la cooperación cara a cara a una red global de colaboración entre regiones, como así también intra-organizaciones. El objetivo de la colaboración se ha convertido también en algo único, ya que los requisitos pasaron de ser físicos a virtuales, y los períodos de operación, más flexibles.

Nebulas no busca subvertir ni excluir otras formas de colaboración; por el contrario, intenta encontrar la forma más apropiada de colaborar y de complementar las formas restantes en nuevos escenarios.

La nueva estructura colaborativa posee las siguientes características:

\begin{itemize}
	\item \textbf{La interacción de la información está mutando de simple a compleja.}

	Las criptodivisas de primera generación (como Bitcoin) registran únicamente información transaccional. Las de segunda generación (como Ethereum) introducen el concepto de contratos inteligentes Turing-completos, con lo cual sus blockchains pasan a ser programables. A partir de ese punto, la interacción creciente entre datos y activos deriva en nuevos problemas y escenarios, como la necesidad de manejar datos dentro y fuera del \blockchain y la interacción entre distintas cadenas.

	\item \textbf{Los roles de usuarios se están incrementando}

	En la comunidad pionera de Bitcoin sólo existían los mineros y los tenedores de activos. Ethereum añadió nuevos grupos a los ecosistemas \blockchain, tales como desarrolladores y usuarios de aplicaciones descentralizadas. A medida que más personas se suman a los ecosistemas \blockchain, la distribución de facultades y responsabilidades a distintos usuarios, de acuerdo a sus roles, se convierte en todo un desafío.

\end{itemize}

Algunos de los problemas actuales son:

\begin{enumerate}
	\item

	\textbf{La gobernanza centralizada no puede lidiar con las situaciones nuevas y complejas}

	Blockchain es esencialmente un sistema autónomo descentralizado basado en un consenso bizantino. Su verdadero atractivo está en su modelo de colaboración abierta basada en un mecanismo de consenso bajo la ideología de la descentralización.~\cite{whitepaper}. No obstante, algunos proyectos \blockchain apuntan a lo contrario, y utilizan la centralización como forma de gobernanza; verbigracia: el arbitraje directo de los casos de \textit{hacking} a través de una \textit{corte de arbitraje central}. La legitimidad e imparcialidad de este enfoque es difícil de garantizar.

	Ante la complejidad de los patrones de interacción de los datos y la ampliación de las funciones de los usuarios, es difícil centralizar un único criterio de evaluación; esto lleva a que los miembros de la comunidad se rebelen. Por ejemplo, el 11 de enero de 2019 las autoridades del proyecto EOS iniciaron una votación para determinar si se finalizaba el \textit{ECAF (EOSIO Core Arbitration Forum)}; el porcentaje de votos afirmativos superó el 98\%~\cite{DeleteECAF}.

	\item

	\textbf{Las reglas existentes de gobernanza descentralizada no son uniformes.}

	En la comunidad bitcoin los usuarios tienen diferentes roles —tales como mineros o tenedores de activos— y cada rol tiene asignada una regla distinta, aunque no es claro quién debe seguir cada regla. Es probable que estos métodos de gobernanza descentralizada causen objetivos de desarrollo comunitario poco claros, lo que dificulta la organización y ejecución efectiva de las actualizaciones (críticas y no-críticas).

	\item

	\textbf{La colaboración descentralizada tradicional es normalmente una tragedia~\cite{TragedyOfTheCommons}.}

	Los proyectos tradicionales de colaboración descentralizada (como lo son un gran número de comunidades de código abierto) tienen modelos de beneficios poco claros y su fuente de financiación se basa a menudo en las donaciones. Las actualizaciones y mejoras dependen demasiado de los intereses de los desarrolladores y se presentan a menudo problemas en la velocidad de evolución del ecosistema debido a las diferentes opiniones en pugna. En la actualidad hay más personas que utilizan los recursos públicos (como el código fuente abierto) que quienes contribuyen a crearlos o mejorarlos. Muchos proyectos de código abierto dependen de grandes corporaciones para recibir donaciones, y a menudo el desarrollo se desvía en la dirección que esas grandes corporaciones exigen, en vez de seguir el curso normal establecido por las necesidades y opiniones de la comunidad; en esencia, se convierten en parte de la corporación.

	Los \textit{tokens} que existen dentro de un ecosistema \blockchain nos dan la oportunidad de resolver el dilema básico de la colaboración descentralizada proporcionando incentivos sostenibles para construir una economía próspera.

	\item

	\textbf{Los incentivos de los mecanismos de consenso en los primeros proyectos \blockchain no son comprensivos y la participación de la comunidad es baja.}

	El sistema de Prueba de Trabajo (PoW~\cite{pow}) que utiliza Bitcoin sólo hace foco en los incentivos para los mineros; este sistema de incentivos únicos no permite el enriquecimiento de todos los usuarios sin importar su rol. Ethereum, con su naturaleza descentralizada, ha recibido reiteradas críticas por la lentitud de su proceso de actualización, debido a que todas las propuestas de mejoras requieren la aprobación de la mayoría de la comunidad y luego su ejecución por parte de operadores de nodos. Esto muestra cuán difícil es unificar las opiniones de un ecosistema entero. Este fenómeno de “no hacer nada” ha llevado a una tasa de participación muy baja en las propuestas de actualización de Ethereum, resultando ello en una implementación tardía y en un daño al desarrollo del ecosistema.

\end{enumerate}

Actualmente no existe una solución perfecta para los problemas descriptos más arriba, y somos conscientes de que en un nuevo mundo de complejidad creciente, la creación de una nueva tecnología que permita solucionar estos escenarios es algo esperado.