\section{链上协作和NAT投票激励}

星云链聚焦链上治理,致力于运用区块链技术来提供更公平的协作环境。

\subsection{链上治理流程}

星云治理通用流程如下:

\begin{enumerate}
\item \textbf{提案阶段}(Proposal Period):发起人在社区公开发起提案,在提案投票期间,通过了NAT链上投票;
\item \textbf{执行阶段}(Develop Period):项目立项,提案人本人或经本人认可的社区成员按计划执行;
\item \textbf{公测阶段}(Testing Period):执行人提交结果,在公测投票期间,通过了NAT链上投票;
\item \textbf{发布阶段}(NBRE Period):通过两次投票的项目,经技术委员会验收核实,且没有争议,即可最终执行发布。
\end{enumerate}

星云链技术开发和运营为项目制,项目的发起、立项、预决算、执行、审核、发布由星云技术委员会负责组织。星云技术委员会受星云理事会委任,只负责监督,保证整个流程公开透明。


\subsection{链上投票}
\subsubsection{基本原则}

星云链生态中的投票将通过星云主网实现,社区投出的每一票,将在星云链的区块上公开、透明的展现。在星云链的系统中,投票将遵循以下基本原则:
\begin{enumerate}
	\item 投票的最基本单位是一个星云主网地址
	\item 星云链的投票权重将参照地址的星云指数
	\item 对系统有积极贡献的行为应被奖励更多的投票权,在投票的场景中,我们认为投票行为是对星云系统有积极贡献的行为,应被激励更多的投票权
	\item 投票唯一的介质为星云指数的资产化表现,Nebulas Autonomous Token (NAT·星元币)
\end{enumerate}

\subsubsection{投票方式}

投票将通过星云链主网上的投票智能合约实现,每个地址可以选择投赞同、反对、弃权三种票。亦可不参与投票。

\subsubsection{投票的唯一介质:Nebulas Autonomous Token (NAT)}

\begin{enumerate}
	\item \textbf{名称}
	Nebulas Autonomous Token
		
	\item \textbf{代号}
	NAT

	\item  \textbf{形式}
	NRC20 token

	\item \textbf{初始发行上限}
	星云指数(Nebulas Rank·NR)是首个衡量区块链多维数据的价值标准,是星云核心的排序算法,也是星云自治元网络中的重要元数据的映射。
	
	NAT是星云指数的资产形态,它将以NRC20 Token的形式体现,并将作为星云生态治理场景中唯一的投票介质。
	
	NAT的发行方式和比特币类似,在总量存在上限的前提下按周期和全网NR的情况以周为单位递减释放。
	
	拥有星云主网地址(黑名单地址除外)的任何用户均可通过参与星云生态获得NAT,NAT协议的数量上限和星云链主网的全网NR值相关,释放量按周递减。递减参数为$λ$,初始参数$λ$=0.997,即约在180周时,发行量递减为第一周的58\% 
	
	NAT的初始发行数量将以2019年5月6日星云主网Nebulas NOVA完成第一次投票升级后的全网NR为参照。在全网NR值不变、初始参数的设定下,NAT的初始数量上限为1,000亿枚。

	\item \textbf{发行方式}
	
	星云主网地址的用户可以通过获得地址的星云指数(Nebulas Rank·NR)、参与星云主网链上投票或者质押星云链主网原生代币NAS三种方式获得NAT。

\textbf{通过获得地址的星云指数(Nebulas Rank·NR)获得NAT}

	对于有NR的地址,NAT协议将按周向该地址进行发放,发放的数量将参照该地址上一周的NR和星云全网的NR值情况。

	每周空投的数量递减,递减的系数为λ。初始情况下$λ$=0.997

	在第$i$周,空投的比例约为 1NR = a*$λ$^$i$ NAT,a= \xpcomment{(插入a的公式及公式中各个参数的定义https://docs.google.com/document/d/11rsqML1Ax_zqru73vP2XULUNGUnHdl2WARAwqd6vkwM/edit)} 

	以2019年4月16日的全网NR情况估算,第一周的空投比例约为1NR = 0.0029NAT

\begin{itemize}

	\item 关于星云指数:
	
	星云指数(Nebulas Rank)是首个衡量区块链多维数据的价值标准,是星云核心的排序算法,也是星云自治元网络中的重要元数据的映射。

	在星云经济体中,治理的基本单位是一个”地址“。星云指数通过对每个地址的贡献度的数学表达,量化了每个“个体”对于经济总量的贡献。在宏观层面,则用经济学中经典的货币数量方程描述了区块链中货币数量、货币价值和流通速率以及生产力几者之间的关系,全网的星云指数可以反映星云生态整体的流通性和活跃度。
	
	“星云指数”分为“核心星云指数”和“扩展星云指数”。在NAT协议的发行中,将主要参考“核心星云指数”。核心星云指数受两个因素影响:1)账户在一定时期内的资产中值 2)账户在一定时期内的出入度衡量。NAT的发行将以周为单位,参考一周之内地址的资产中值和出入度衡量计算所得的星云指数。

	关于星云指数的更多信息,请参考2018年6月星云研究院发表的《星云指数黄皮书1.0.3》

	\item 如何查询星云指数:
	星云指数将在2019年5月6日,星云Nebulas NOVA完成第一次投票升级后上链,后续可以通过星云Nebulas NOVA的核心能力Nebulas Blockchain Runtime Environment(NBRE)实现升级。
查询你的核心星云指数,请点击:https://nr.nebulas.io/nr/#/

\end{itemize}

\textbf{通过质押星云链主网原生代币NAS获得NAT}

	从2019年5月6日开始,星云主网地址用户可以选择向投票的智能合约质押星云主网原生代币NAS获得NAT。

	质押NAS的用户将从质押开始后的第2周(即2019年5月13日)开始获得NAT的空投,用户取回质押则停止获得空投。

	每周获得空投的数量递减,递减的系数为λ,初始情况下λ=0.997

	\textbf{质押NAS获得NAT的数量}
	
		第$i$周质押NAS获得NAT的比例为 x NAS = $α$*b*c*$λ$^$i$ NAT
		其中$α$为质押系数

	\xpcomment{(插入a的公式及公式中各个参数的定义https://docs.google.com/document/d/1JayzvyownGsEPuYSW0KjS4FU_pA7OW96wq3Cfr0RArM/edit)} 

	\textbf{进行和取消质押}

	\begin{itemize}
		\item 发起质押:用户可以通过使用星云钱包NAS Nano Pro或其他支持星云主网币的客户端向投票智能合约发送交易,确认要质押的NAS的数量
		\item 停止质押:用户如想停止质押,可以通过NAS Nano Pro或其他客户端调用智能合约申请取消,取消后可立即取回质押的NAS,取消质押后,则停止获得NAT空投。
		\item 质押的地址:为了保证用户可以获得并管理NAT,用户需要用自己掌握私钥的主网地址向星云的投票智能合约发送NAS完成质押,请勿使用交易所账户发送交易。
	\end{itemize}

\textbf{通过参与星云主网链上投票获得NAT}

	每周星云主网的地址会获得了NAT空投后,可以选择投赞同、反对、弃权三种票,亦可不参与投票。用户投出赞成、反对和弃权票均可获得投票激励。如在此次空投周期中,用户未参与投票,则无法获得激励。

	\textbf{激励的规模}
		我们认为激励的规模应该是适当的,不应该被恶意使用的。所以投票的激励会取决于:
		1)地址当周投出的NAT的数量
		2)地址上一周的NR值及该NR值对应可获得的NAT
		
		用户投出符合自己地址NR值对应的NAT会得到激励,如果用户投出的超过自己的NR值的部分对应的NAT,超出的部分无法获得激励。
		
		第一周,地址获得的激励的规模为 $μ$*min(NATv,NATnr)

		其中:
		$μ$为激励系数,初始参数下$μ$=10
		NATv为当周地址投出的NAT数额
		NATnr为地址上一周的NR值对应可以获得的NAT

		即当地址当周投出的NATv小于或等于NATnr的时候,他获得的激励数量为NATv,当地址当周投出的NATv大于NATnr的时候,他获得的激励的数量为 μ*NATnr。

		举例来说,假如某个地址根据上一周NR值获得了10NAT的空投,地址上一共有1000NAT,在当周假如此地址投出10NAT,获得100NAT投票激励。假如此地址投出1000NAT,仍获得100NAT激励。

		和空投部分及质押部分获得的NAT一样,投票激励的NAT也是按周递减的,递减系数为λ。

\textbf{NAT的使用场景}

NAT将作为星云生态中投票场景的唯一介质,支持星云推进社区的治理。社区成员可以通过使用NAT投票表达自己自己对星云生态的意见,包括但不限于星云理事会的选举、星云主网NBRE中执行的IR及星云项目提案的立项表决等。

\textbf{管理}

用户可以在NAS Nano Pro及其他支持NRC20的客户端中管理自己的NAT。同时,用户可以在支持星云链的区块链浏览器上(比如explorer.nebulas.io)查看NAT的交易、流通情况等。

\textbf{投入投票智能合约的NAT}在每个发行周期内,用户投入星云链投票智能合约的NAT将被立即烧毁,烧毁的比例会按照周期递减,递减速率和NAT发行的递减速率一致。在每个周期内,未被烧毁的NAT,将会在每个周期的第二次空投的过程中返还给本周期内参与投票的用户。每个周期内烧毁部分的NAT将按照一下函数计算:
\xpcomment{(插入NAT烧毁速率的函数)}

\textbf{投票的手续费}
每此投票将收取3\%的NAT将作为投票手续费,此部分手续费由星云理事会授权交由基金会作为NAT项目的专项运营资金管理,项目团队不得将此部分手续费直接用于投票。

\item \textbf{NAT的黑名单地址}
在星云治理的基本概述中, 我们提出了星云生态治理的三点基本主张。

NAT是星云生态中治理中的重要投票介质,在NAT的发行过程中,不符合星云治理基本原则的地址将在NAT的发行中被归为黑名单地址。黑名单地址只能根据地址享有的权利的情况获得部分NAT。

如中心化的交易所地址。

依照星云地址第一基本权利,该地址具备拥有和操作星云链上资产的权利。所以交易所的归集地址是可以在同等条件下通过获得地址的NR获得NAT,但这部分NAT的产权属于相应的交易所用户。

依照星云地址第二、第三基本权利,在交易所证明该归集地址充分代表了相应托管资产用户提案和投票意愿之前,交易所归集地址并不具备发起提案和参与提案投票的权利,因此亦不能通过参与投票获得投票部分的NAT激励。

\item \textbf{NAT参数的调整}

	NAT的发行过程涉及到如下系数:
	$α$:质押系数,初始数值$α$=5
	$μ$:投票奖励系数,初始数值$μ$=10
	$λ$:递减系数,初始数值$λ$=10

	系数的调整需要经过星云生态的治理投票流程,星云基金会或NAT的项目团队无权擅自调整系数。

\subsubsection{投票规则}

\begin{itemize}

	\item \textbf{投票的监督}:公开投票由技术委员会负责组织和管理。公开投票接受所有人的公开监督。针对违反社区治理基本准则的提案,技术委员会可以向理事会发起重审提案申请。星云理事会作为星云社区治理的监管机构,有权对任何提案的投票发起二次投票。对于同一提案,理事会可以且仅可以发起一次二次投票。
	
	\item \textbf{项目投票通过的标准}
	
	投票是否通过将通过两个维度的标准来衡量: 1) 投票的参与度 2)赞成票的占比

	1) 投票的参与度:

	对于涉及到使用公共资产支持的项目,投票的参与度不得低于该项目提起的资产的金额占全网的的比例。
	如某提案要求动用X个NAS支持,此时星云主网中流通的NAS为Y。(此处定义的流通的NAS为任何未在锁仓/质押状态的,可随时在星云主网上进行转账交易的NAS)
	则此项目通过需要达成的全网的投票参与度不得低于X/Y,换算成NAT来表示,及参与到此次项目中投票的NAT与该周期初期空投给用户的NAT的比例不得低于X/Y

	对于不涉及到使用公共资产支持的项目,投票的参与度不得低于51\%,此类项目包括但不限于星云主网参数的调整、NBRE要执行的NPR等

	2)赞成票的占比
	在满足投票最低参与度之外,某一项目投票通过还需要满足赞成票占总投入票数的比例不得低于51\%
	即假设某一项目共收到票数为N,其中赞成票为Y,反对票为N,弃权票为A,则只有当Y/(Y+N+A) >= 51\%时,此项目才被视为投票通过

	\item \textbf{二次投票}
	
	根据前文所述,星云理事会作为星云生态中治理流程正当性的监督者,有权对某一项目提起且仅能提起一次进行“二次投票”的要求。
	当理事会提出“二次投票”的要求时,该项目被视为进入到新的投票周期进行一次新的投票。第一次投票过程中的结果不被执行,第一次投票投出的NAT不予返还,会按照当周期的烧毁速率进行烧毁。
	在二次投票的过程中,该项目投票的参与度需大于第一次投票的参与度。即假设第一次投票的参与度为X/Y,则第二次投票的参与度应大于X/Y,且赞成票的比例不低于51\%方可视为投票通过。
	
\end{itemize}

<<<<<<< HEAD
\subsection{执行}
\textbf{提案人义务}
\begin{enumerate}
\item 监督执行人执行,各时间点没有错失。
\item 维护社区讨论。
\end{enumerate}

\textbf{提案人权利}
\begin{enumerate}
\item 享有提案中涉及的应得回报。
\item 在申请参与执行的社区用户中挑选合适的人来执行。
\end{enumerate}

\textbf{执行人义务}
\begin{enumerate}
\item 定期公示进度,保证执行过程公开透明。
\item 保证按时保质完成,并按照提案要求提交成果。
\end{enumerate}

\textbf{执行人权利}
\begin{enumerate}
\item 享有提案中涉及的应得回报。
\item 在不影响进度并且能达成最终目标的情况下,有权调整具体的执行步骤。
\end{enumerate}

\textbf{执行时长}

考虑到币价波动,建议不超过一季度。复杂度高的项目建议拆分阶段进行。

\textbf{提前终止}

如果开发没有按时完成,提案立即提前宣告失败。社区预留资金不会发放。

\subsection{公测}
\textbf{公测步骤}

完整的公测需要经过以下3个步骤:
\begin{enumerate}
\item 提交项目成果。开始公开测试。
\item 投票。二次投票方法和时长与一次投票相同,NAT占比y\%应大于等于x\%。
\item 技术委员会审核。通过二次投票的技术类提案,会随机派给三名技术委员会相关技术人员进行复审,全部通过即视为通过,进入发布流程。
\end{enumerate}

\textbf{测试时长}

技术类项目按照开发时长折算,最短时间为开发周期的10\%。

\subsection{发布}
非技术项目通过公测后即视为成功。技术项目需要进行链上数据更新。

\textbf{链上数据更新流程}

当协议需要升级之时,包含升级的代码会被包含在一个特殊的交易当中。这类交易只能由星云理事会暂时代为控制的主节点发起,并广播到全网。之后所有节点一旦接收到这类交易,NBRE将会自动执行其中包含的升级代码,完成更新。与此同时,核心协议的旧版本仍会在链上,以作参考。注意,节点拒绝更新只能重写本地代码。

\textbf{技术相对优势}
\begin{enumerate}
	\item 其他区块链项目升级主要还是通过硬/软分叉的方式,星云升级不需要进行分叉。
\item 交易所等不需要充停。
\item 高效率,零成本。
\end{enumerate}

举例:每个区块是火车上的一节车厢,那么每个区块中的交易就相当于车厢中的乘客。之前,如果火车发生故障,我们需要把它送到工厂维修,就像是我们手动修复漏洞,升级代码。当主网发现漏洞之后,这时,火车不能继续运行,车上的乘客也会因此受到影响。而NBRE的升级方式相当于把零部件提前放在火车上,用一节车厢专门储存零部件,当火车发生故障之时,车上的工人就可以及时换上新的部件。在此过程中,火车仍可正常运作。
=======
>>>>>>> 0287099a1c014b60b6a99759cc9c66fa8e9a73aa



