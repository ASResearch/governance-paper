\section{链上协作和NAT投票激励}

星云链聚焦链上治理,致力于运用区块链技术来提供更公平的协作环境。

\subsection{链上治理流程}

星云治理通用流程如下:

\begin{enumerate}
\item \textbf{提案阶段}(Proposal Period):发起人在社区公开发起提案,在提案投票期间,通过了NAT链上投票;
\item \textbf{执行阶段}(Develop Period):项目立项,提案人本人或经本人认可的社区成员按计划执行;
\item \textbf{公测阶段}(Testing Period):执行人提交结果,在公测投票期间,通过了NAT链上投票;
\item \textbf{发布阶段}(NBRE Period):通过了两次投票的项目,社区没有异议且经技术委员会验收核实,即可最终执行发布。
\end{enumerate}

星云技术委员会受星云理事会委任,负责监督治理流程,保证整个流程公开透明。


\subsection{投票基本原则}

星云链生态中的投票将通过星云主网实现,社区投出的每一票,将在星云链的区块上公开、透明的展现。在星云链的系统中,投票将遵循以下基本原则:

\begin{enumerate}
	\item 投票的最基本单位是一个星云主网地址。
	\item 星云链的投票权重将参照地址的星云指数。
	\item 对系统有积极贡献的行为应被奖励更多的投票权,在投票的场景中,我们认为投票行为是对星云系统有积极贡献的行为,应被激励更多的投票权。
	\item 投票唯一的介质为星云指数的资产化表现:NAT(Nebulas Autonomous Token)。
\end{enumerate}

\subsection{投票方式}

投票将通过星云链主网上的投票智能合约实现,每个地址可以选择投赞同、反对、弃权三种票。亦可不参与投票。

\subsection{投票唯一介质(NAT)概述}

\textbf{名称:} Nebulas Autonomous Token

\textbf{代号:} NAT

\textbf{形式:} NRC20 token

星云指数是首个衡量区块链多维数据的价值标准,是星云核心的排序算法,也是星云自治元网络中的重要元数据的映射。
	
NAT是星云指数的资产形态,它将以NRC20 Token的形式体现,并将作为星云生态治理场景中唯一的投票介质。

\textbf{NAT发行:}
	
NAT的发行方式和比特币类似,在总量存在上限的前提下按周期和全网NR的情况以周为单位递减释放。

\textbf{初始发行上限:}

拥有星云主网地址(黑名单地址除外)的任何用户均可通过参与星云生态获得NAT,NAT协议的数量上限和星云链主网的全网NR值相关,释放量按周递减。递减参数为:$\lambda$。初始递减参数$\lambda$=0.997,即约在第180周时,发行量递减为第一周的58\%。

NAT的初始发行数量将以2019年5月6日星云主网Nebulas NOVA\footnote{https://nebulas.io/cn/nova.html}完成第一次投票升级后的全网NR为参照。在全网NR值不变、初始参数的设定下,NAT的初始数量上限为1,000亿枚。

\textbf{使用场景:}

NAT将作为星云生态中投票场景的唯一介质,支持星云推进社区的治理。社区成员可以通过使用NAT投票表达自己自己对星云生态的意见,包括但不限于星云理事会的选举、星云主网NBRE中执行的IR及星云项目提案的立项表决等。


\subsection{获取投票介质}

星云主网地址的用户可以通过提升地址的星云指数、参与星云主网链上投票,或质押星云链主网原生代币星云币(NAS)三种方式获得NAT。


\subsubsection{通过提升地址的星云指数获得NAT}

对于有NR的地址,NAT协议将按周向该地址进行发放,发放的数量将参照该地址上一周的NR和星云全网的NR值情况。

每周空投的数量递减,递减系数为$\lambda$。初始情况下:$\lambda$=0.997

在第$i$周,空投的比例约为:

$$1 NR = a\lambda^{i} NAT$$

$$a=\frac{10^{11}(1-\lambda)}{{(\mu +1)_{}^{\chi}\textrm{NE}}+{_{}^{\chi}\textrm{E}}}$$    

其中:
\begin{itemize}
	\item $\lambda$:衰减系数;
	\item $\mu$:投票行为的激励参数;
	\item $_{}^{\chi}\textrm{NE}$:全网非交易所地址的NR的总和;
	\item $_{}^{\chi}\textrm{E}$:全网交易所地址的NR总和
\end{itemize}

以2019年4月16日的全网NR情况估算,第一周的空投比例约为:

$$1 NR = 0.0029 NAT$$

\textbf{关于星云指数:}
	
星云指数是首个衡量区块链多维数据的价值标准,是星云核心的排序算法,也是星云自治元网络中的重要元数据的映射。

在星云经济体中,治理的基本单位是一个”地址“。星云指数通过对每个地址的贡献度的数学表达,量化了每个“个体”对于经济总量的贡献。在宏观层面,则用经济学中经典的货币数量方程描述了区块链中货币数量、货币价值和流通速率以及生产力几者之间的关系,全网的星云指数可以反映星云生态整体的流通性和活跃度。

“星云指数”分为“核心星云指数”和“扩展星云指数”。在NAT协议的发行中,将主要参考“核心星云指数”。核心星云指数受两个因素影响:

\begin{enumerate}
	\item 账户在一定时期内的资产中值。
	\item 账户在一定时期内的出入度衡量。NAT的发行将以周为单位,参考一周之内地址的资产中值和出入度衡量计算所得的星云指数。
\end{enumerate}

关于星云指数的更多信息,请参考2018年6月星云研究院发表的《星云指数黄皮书》\footnote{1.0.3版本:https://nebulas.io/docs/NebulasYellowpaperZh.pdf}。

\textbf{查询星云指数:}

星云指数在2019年5月6日星云Nebulas NOVA完成第一次投票升级后上链,可以通过星云Nebulas NOVA的核心能力星云区块链可执行环境(Nebulas Blockchain Runtime Environment,NBRE)实现升级。核心星云指数开源,可在线查询\footnote{https://nr.nebulas.io/nr/}。


\subsubsection{通过质押星云链主网原生代币NAS获得NAT}

从2019年5月6日开始,星云主网地址用户可以选择向投票的智能合约质押星云主网原生代币星云币(NAS)获得NAT。

质押NAS的用户将从质押开始后的第2周(即2019年5月13日)开始获得NAT的空投,用户取回质押则停止获得空投。

每周获得空投的数量递减,递减的系数为$\lambda$,初始情况下$\lambda$=0.997

\textbf{质押NAS获得NAT的数量}

第$i$周质押NAS获得NAT的比例为: 

$$x NAS = \alpha*b*c*\lambda^{i} NAT$$

其中:

\begin{itemize}
	\item $x$:质押NAS的数量;
	\item $\alpha$:质押系数,初始状态$\alpha$=5;
	\item $b$:NR和NAT的兑换比例,即:

	$$b=\frac{10^{11}(1-\lambda)}{{(\mu +1)_{}^{\chi}\textrm{NE}}+{_{}^{\chi}\textrm{E}}}$$

	\item $$c=\frac{x}{1+\sqrt{\frac{200}{x}}}$$
\end{itemize}


\textbf{发起质押:} 
	
用户可以通过使用星云钱包NAS Nano Pro或其他支持星云主网币的客户端向投票智能合约发送交易,确认要质押的NAS的数量。

\textbf{取消质押:}

用户可以通过NAS Nano Pro或其他客户端调用智能合约申请取消,取消后可立即取回质押的NAS,取消质押后,则停止获得NAT空投。

\textbf{质押地址:}

为了保证用户可以获得并管理NAT,用户需要用自己掌握私钥的主网地址向星云的投票智能合约发送NAS完成质押,请勿使用交易所账户发送交易。


\subsubsection{通过参与星云主网链上投票获得NAT}

星云主网地址获得了NAT空投后,可以选择参与或不参与社区治理链上投票。如在此次空投周期中,用户参与了投票,无论投出赞成、反对还是弃权票,均可获得投票激励。如果未参与投票,则无法获得激励。

\textbf{激励的规模:}

我们认为激励的规模应该是适当的,不应该被恶意使用的。所以投票的激励会取决于:

\begin{enumerate}
	\item 该地址当周投出的NAT的数量。
	\item 该地址上一周的NR值及该NR值对应可获得的NAT。
\end{enumerate}

当用户投出符合自己地址NR值对应的NAT会得到激励,如果用户投出的超过自己的NR值的部分对应的NAT,超出的部分无法获得激励。
	
第一周,地址获得的激励的规模为:

$$\mu*min(NATv, NATnr)$$

其中:

\begin{itemize}
	\item $\mu$:激励系数,初始参数$\mu$=10;
	\item $NATv$:当周地址投出的NAT数额;
	\item $NATnr$:地址上一周的NR值对应可以获得的NAT。即,当地址当周投出的$NATv$小于或等于$NATnr$的时候,获得的激励数量为$NATv$,当地址当周投出的$NATv$大于$NATnr$的时候,他获得的激励数量为$\mu*NATnr$。

举例来说,假如某个地址根据上一周NR值获得了10 NAT的空投,地址上一共有1,000 NAT,在当周假如此地址投出10 NAT,获得100 NAT投票激励。假如此地址投出1,000 NAT,仍获得100 NAT激励。

和空投部分及质押部分获得的NAT一样,投票激励的NAT也是按周递减的,递减系数为$\lambda$。

\end{itemize}

\subsection{管理NAT}

用户可以在NAS Nano Pro及其他支持NRC20的客户端中管理自己的NAT。同时,用户可以在支持星云链的区块链浏览器\footnote{https://explorer.nebulas.io}上查看NAT的交易、流通情况等。

\textbf{投入投票智能合约的NAT}

在每个发行周期内,用户投入星云链投票智能合约的NAT将被立即烧毁,烧毁的比例会按照周期递减,递减速率和NAT发行的递减速率一致。在每个周期内,未被烧毁的NAT,将会在每个周期的第二次空投的过程中返还给本周期内参与投票的用户。每个周期内烧毁部分的NAT将按照一下函数计算:

\xpcomment{(插入NAT烧毁速率的函数)}

\textbf{投票的手续费}

每此投票将收取3\%的NAT将作为投票手续费,此部分手续费由星云理事会授权交由基金会作为NAT项目的专项运营资金管理,项目团队不得将此部分手续费直接用于投票。

\textbf{NAT的黑名单地址}

在星云治理的基本概述中, 我们提出了星云生态治理的三点基本主张。

NAT是星云生态中治理中的重要投票介质,在NAT的发行过程中,不符合星云治理基本原则的地址将在NAT的发行中被归为黑名单地址。黑名单地址只能根据地址享有的权利的情况获得部分NAT。如中心化的交易所地址。

依照星云地址第一基本权利,该地址具备拥有和操作星云链上资产的权利。所以交易所的归集地址是可以在同等条件下通过获得地址的NR获得NAT,但这部分NAT的产权属于相应的交易所用户。

依照星云地址第二、第三基本权利,在交易所证明该归集地址充分代表了相应托管资产用户提案和投票意愿之前,交易所归集地址并不具备发起提案和参与提案投票的权利,因此亦不能通过参与投票获得投票部分的NAT激励。

\textbf{NAT参数调整}

NAT的发行过程涉及到如下系数:

\begin{enumerate}
	\item $\alpha$:质押系数,初始数值$\alpha$=5
	\item $\mu$:投票奖励系数,初始数值$\mu$=10
	\item $\lambda$:递减系数,初始数值$\lambda$=10
\end{enumerate}

系数的调整需要经过星云生态的治理投票流程,星云基金会或NAT的项目团队无权擅自调整系数。

\subsection{投票规则}
	
\subsubsection{投票通过标准}
	
投票是否通过将通过两个维度的标准来衡量:投票的参与度和赞成票的占比。

\textbf{投票的参与度:}

对于涉及到使用公共资产支持的项目,投票的参与度不得低于该项目提起的资产的金额占全网的的比例。

如某提案要求动用X个NAS支持,此时星云主网中流通的NAS\footnote{此处定义流通NAS为任何未在锁仓/质押状态、可随时在星云主网上进行转账交易的NAS。}为Y。

则此项目通过需要达成的全网的投票参与度不得低于X/Y,换算成NAT来表示,及参与到此次项目中投票的NAT与该周期初期空投给用户的NAT的比例不得低于X/Y。

对于不涉及到使用公共资产支持的项目,投票的参与度不得低于51\%,此类项目包括但不限于星云主网参数的调整、NBRE要执行的NPR等。

\textbf{赞成票的占比:}

在满足投票最低参与度之外,某一项目投票通过还需要满足赞成票占总投入票数的比例不得低于51\%。

即假设某一项目共收到票数为N,其中赞成票为Y,反对票为N,弃权票为A,则只有当Y/(Y+N+A) >= 51\%时,此项目才被视为投票通过。

\subsection{投票的监督}

公开投票由技术委员会负责组织和管理。公开投票接受所有人的公开监督。针对违反社区治理基本准则的提案,技术委员会可以向理事会发起重审提案申请。星云理事会作为星云社区治理的监管机构,有权对任何提案的投票发起二次投票。对于同一提案,理事会可以且仅可以发起一次二次投票。

根据前文所述,星云理事会作为星云生态中治理流程正当性的监督者,有权对某一项目提起且仅能提起一次进行“二次投票”的要求。

当理事会提出“二次投票”的要求时,该项目被视为进入到新的投票周期进行一次新的投票。第一次投票过程中的结果不被执行,第一次投票投出的NAT不予返还,会按照当周期的烧毁速率进行烧毁。

在二次投票的过程中,该项目投票的参与度需大于第一次投票的参与度。即假设第一次投票的参与度为X/Y,则第二次投票的参与度应大于X/Y,且赞成票的比例不低于51\%方可视为投票通过。




