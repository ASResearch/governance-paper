\section{治理流程}
在NBRE中,升级通过提交IR来进行。不同IR的提交会对星云链整体系统环境的性能产生实质性的影响,IR的提交实际程度上决定了星云链主网的升级和性能推进方向。

现阶段星云链的功能特点并未全部实现,星云团队会慎重开放提交IR的权限。当前治理流程如下:

\textbf{治理流程(现阶段)}	
\begin{enumerate}
\item \textbf{提案阶段}(Proposal Period):在社区公开讨论提案,发起项目,并可申请使用社区预留资金;
\item \textbf{投票阶段}(Proposal Voting Period):投票区块期间,赞同票NR占比超过全网NR值一定比例x\%,视作通过;
\item \textbf{开发阶段}(Develop Period):提案人进行开发,或在社区寻找开发者共同开发,开发过程在协作平台(go.nebulas.io)上公开;
\item \textbf{测试阶段}(Testing Period):经主网团队操作,部署测试网公测,关于IR是否可以部署到主网进行链上加权投票,投票区块期间,赞同票NR占比超过全网NR值不小于x\%,视作通过;
\item \textbf{发布阶段}(NBRE Period):经技术委员会验收核实,且测试期内没有争议,IR被NBRE执行,并入主网。
\end{enumerate}

\textbf{治理流程(NF实现后)}

NF实现后,任何人可以提交IR,以主网NF侧链形式存在,不需要经过主网团队审批即可自行测试。

\subsection{提案}
星云链技术开发和运营为项目制,项目的发起、立项、预决算、执行、审核由星云技术委员会负责组织。星云技术委员会只负责监督,保证整个流程公开透明。如涉及线下活动,应符合当地相关法律法规。原则上星云技术委员会不介入其中,并且不承担任何责任。

\textbf{提案人}

所有社区成员。

\textbf{提案方式}

在提案专区公开发出提案。

\textbf{提案内容}

一份完整的提案需要包括以下内容:
\begin{enumerate}
\item 提案人姓名(需要登录)
\item 提案描述(当前问题、解决方案、理由)
\item 提案成果形式(如github地址、DApp、PDF、文稿链接等,也可以是综合性结果或报告)
\item 指定执行人,无指定执行人则可以在社区公开招募执行人
\item 各时间点设定:投票开始和结束时间、招募执行人时间、执行阶段时间点和截止日期、执行耗时预估、测试时长
\item 如果申请使用社区预留资金,需要填写明确金额(NAS作为计价单位)
\end{enumerate}

\textbf{提案时长}

由提案人自由定义。

\textbf{提案修改}

在投票正式开始之前,提案人可以修改提案所有内容。其他人可以评论、关注等。每次修改需要支付$n^2$ NAS。这部分奖励会均分给当次提案前参与过讨论的用户。

\textbf{提案取消}

在正式投票开始之前,提案人可以取消提案。正式投票开始后,不可取消提案。

\subsection{投票}
\subsubsection{投票的基本原则}

星云链聚焦链上治理,致力于运用区块链技术来提供更公平的协作环境。星云链生态中的投票将通过星云主网实现,社区投出的每一票,将在星云链的区块上公开、透明的展现。

同时星云链期望构建自治元网络,希望借助网络中的元数据和元数据的超映射,处理复杂的数据和交互及协作关系。星云链的投票权重将参照星云自治元网络中的元数据及元数据的超映射。

在星云链的系统中,投票将遵循以下原则:
\begin{enumerate}
	\item 投票的最基本单位是一个星云主网地址
	\item 星云链的投票权重将参照地址的星云指数
	\item 对系统有积极贡献的行为应被奖励更多的投票权,在投票的场景中,我们认为投票行为是对星云系统有积极贡献的行为,应被激励更多的投票权
	\item 投票唯一的介质为星云指数的资产化表现,Nebulas Autonomous Token (NAT·星元币)
\end{enumerate}

\subsubsection{投票方式}

投票将通过星云链主网上的投票智能合约实现,每个地址可以选择投赞同、反对、弃权三种票。亦可不参与投票。

\subsubsection{投票的唯一介质:Nebulas Autonomous Token (NAT·星元币)}

\begin{enumerate}
	\item \textbf{名称}
		星元币 Nebulas Autonomous Token
		
	\item \textbf{代号}
	NAT

	\item  \textbf{形式}
	NRC20 token

	\item \textbf{发行量}
	NAT的发行量与星云链全网的星云指数(Nebulas Rank,NR)相关。在星云链全网星云指数恒定的情况下,NAT发行量恒定且释放的数量按周期递减,衰减系数为$\lambda$。
	
	NAT的初始发行总量将参照星云链主网Nebulas NOVA上线时的全网星云指数,总发行量为【1000】亿枚,初始设定$\lambda=0.99$, 在$\lambda$和全网NR指数不变的情况下,NAT的释放量在第二年衰减为前一年的60\%。 \xpcomment{(待确认—)} 

\xpcomment{(关于NAT的具体的发行的递减参数请参见【5.3】NAT算法)} 

	\item \textbf{发行方式}
	
	NAT将通过空投和质押星云链主网原生代币NAS两种方式,针对且只针对星云主网的用户发行。
	
	NAT的发行将分周期进行,定义每个发行周期为$i$,【且发行周期和投票周期一致】,NAT的发行将遵循如下原则:

		\begin{itemize}
			\item 每个空投周期内单位地址获得的NAT和该地址的NR正相关,下一个周期空投的比例在前个周期的基础上进行衰减,衰减系数为$\lambda$
			\item 地址在当前周期的投票行为会对当前周期空投的NAT数量有额外加成,加成系数为:$\mu$
			\item 用户质押星云主网的原生资产NAS达到6个周期以上,可以获得NAT
			\item 空投部分和质押部分不可同时获得
		\end{itemize}
\end{enumerate}

\textbf{空投部分}

在每个发行周期内,持有星云主网的地址,且在过去6个空投周期内的NR加权不为零的地址,可以获得NAT空投。

在单一的发行周期内将进行两次空投,
1. 在空投周期初始,空投比例将参照以下函数,使得空投的比例和用户地址的NR值成正比,且空投比例会随着时间进行衰减。在越靠前的周期内,同一地址获得的空投的比例越高。
\xpcomment{(插入A部分的函数)} 

2. 在空投周期结束前,按照用户在此周期内的投票行为进行第二次空投,此次空投将给予参与投票的用户额外的NAT加成。用户投出赞成、反对和弃权票均可获得加成。如此次空投周期中,用户未参与投票,则无法获得此部分加成。加成部分按照一下函数计算:
\xpcomment{(插入B部分的函数)} 

\textbf{质押部分}

用户向投票的智能合约质押NAS可以获得一定比例的NAT返还,用户将按照自己质押NAS占全网总质押NAS的比例瓜分质押部分的NAT,在每个发行周期内,质押部分返还的NAT将按照一下函数计算:
\xpcomment{(插入C部分的函数)} 

\textbf{NAT的使用场景}

NAT将作为星云生态中投票场景的唯一介质,支持星云推进社区的治理。社区成员可以通过使用NAT投票表达自己自己对星云生态的意见,包括但不限于星云理事会的选举、星云主网NBRE中执行的IR及星云项目提案的立项表决等。

\textbf{管理}

用户可以在NAS Nano Pro及其他支持NRC20的钱包中管理自己的NAT。同时,用户可以在支持星云链的区块链浏览器上(比如explorer.nebulas.io)查看NAT的交易、流通情况等。

\textbf{NAT的黑名单地址}
目前,考虑到中心化交易所的地址上的资产为代为用户管理的资产,在当前的技术条件下,无法判断投票行为是否可以真实反映用户的意愿。因此目前,在星云生态的投票场景中,中心化交易所的地址将作为黑名单地址处理,不具备投票权亦不能享受投票行为的加成部分。但是中心化交易所的地址作为星云主网的地址仍具有和其他地址一样的使用和管理资产的权利,仍可以获得NAT的空投。

\textbf{投入投票智能合约的NAT}在每个发行周期内,用户投入星云链投票智能合约的NAT将被立即烧毁,烧毁的比例会按照周期递减,递减速率和NAT发行的递减速率一致。在每个周期内,未被烧毁的NAT,将会在每个周期的第二次空投的过程中返还给本周期内参与投票的用户。每个周期内烧毁部分的NAT将按照一下函数计算:
\xpcomment{(插入NAT烧毁速率的函数)}

\textbf{投票的手续费}
每期投出的NAT中的1\%将作为投票手续费,此部分手续费为星云生态中的公共资产。由社区共同监管,并由星云理事会监督此部分资产使用的正当性。星云理事会需要将此部分资产公示并定期向社区公示资产的使用情况。星云理事会不具有手续费的所有权,亦不能将手续费直接用于投票。

\subsubsection{投票规则}

\begin{itemize}
	\item \textbf{投票的监督}:公开投票由技术委员会负责组织和管理。公开投票接受所有人的公开监督。针对违反社区治理基本准则的提案,技术委员会可以向理事会发起重审提案申请。星云理事会作为星云社区治理的监管机构,有权对任何提案的投票发起二次投票。对于同一提案,理事会可以且仅可以发起一次二次投票。
	\item \textbf{项目投票通过的标准}
	
	投票是否通过将通过两个维度的标准来衡量: 1) 投票的参与度 2)赞成票的占比

	1) 投票的参与度:

	对于涉及到使用公共资产支持的项目,投票的参与度不得低于该项目提起的资产的金额占全网的的比例。
	如某提案要求动用X个NAS支持,此时星云主网中流通的NAS为Y。(此处定义的流通的NAS为任何未在锁仓/质押状态的,可随时在星云主网上进行转账交易的NAS)
	则此项目通过需要达成的全网的投票参与度不得低于X/Y,换算成NAT来表示,及参与到此次项目中投票的NAT与该周期初期空投给用户的NAT的比例不得低于X/Y

	对于不涉及到使用公共资产支持的项目,投票的参与度不得低于51\%,此类项目包括但不限于星云主网参数的调整、NBRE要执行的NPR等

	2)赞成票的占比
	在满足投票最低参与度之外,某一项目投票通过还需要满足赞成票占总投入票数的比例不得低于51\%
	即假设某一项目共收到票数N,其中赞成票数为Y,反对票数为N,弃权票数为A,则只有当Y/(Y+N+A) >= 51\%时,此项目才被视为投票通过

	\item \textbf{二次投票}
	
	根据前文所述,星云理事会会作为星云生态中治理流程正当性的监督者,有权且仅有权对某一项目提起一次“二次投票”的要求。当理事会发起“二次投票”的要求时,该项目被视为进入到新的投票周期进行一次新的投票。第一次投票过程中的NAT不予返还,会按照当周期的烧毁速率进行烧毁。
	在二次投票的过程中,该项目投票的参与度需大于第一次投票的参与度。即假设第一次投票的参与度为X/Y,则第二次投票的参与度应大于X/Y,且赞成票的比例不低于51\%方可视为投票通过。
	 
\end{itemize}

\textbf{}



\subsection{执行}
\textbf{提案人义务}
\begin{enumerate}
\item 监督执行人执行,各时间点没有错失。
\item 维护社区讨论。
\end{enumerate}

\textbf{提案人权利}
\begin{enumerate}
\item 享有提案中涉及的应得回报。
\item 在申请参与执行的社区用户中挑选合适的人来执行。
\end{enumerate}

\textbf{执行人义务}
\begin{enumerate}
\item 定期公示进度,保证执行过程公开透明。
\item 保证按时保质完成,并按照提案要求提交成果。
\end{enumerate}

\textbf{执行人权利}
\begin{enumerate}
\item 享有提案中涉及的应得回报。
\item 在不影响进度并且能达成最终目标的情况下,有权调整具体的执行步骤。
\end{enumerate}

\textbf{执行时长}

考虑到币价波动,建议不超过一季度。复杂度高的项目建议拆分阶段进行。

\textbf{提前终止}

如果开发没有按时完成,提案立即提前宣告失败。社区预留资金不会发放。

\subsection{公测}
\textbf{公测步骤}

完整的公测需要经过以下3个步骤:
\begin{enumerate}
\item 提交项目成果。开始公开测试。
\item 投票。二次投票方法和时长与一次投票相同,NAT占比y\%应大于等于x\%。
\item 技术委员会审核。通过二次投票的技术类提案,会随机派给三名技术委员会相关技术人员进行复审,全部通过即视为通过,进入发布流程。
\end{enumerate}

\textbf{测试时长}

技术类项目按照开发时长折算,最短时间为开发周期的10\%。

\subsection{发布}
非技术项目通过公测后即视为成功。技术项目需要进行链上数据更新。

\textbf{链上数据更新流程}

当协议需要升级之时,包含升级的代码会被包含在一个特殊的交易当中。这类交易只能由星云理事会暂时代为控制的主节点发起,并广播到全网。之后所有节点一旦接收到这类交易,NBRE将会自动执行其中包含的升级代码,完成更新。与此同时,核心协议的旧版本仍会在链上,以作参考。注意,节点拒绝更新只能重写本地代码。

\textbf{技术相对优势}
\begin{enumerate}
	\item 其他区块链项目升级主要还是通过硬/软分叉的方式,星云升级不需要进行分叉。
\item 交易所等不需要充停。
\item 高效率,零成本。
\end{enumerate}

举例:每个区块是火车上的一节车厢,那么每个区块中的交易就相当于车厢中的乘客。之前,如果火车发生故障,我们需要把它送到工厂维修,就像是我们手动修复漏洞,升级代码。当主网发现漏洞之后,这时,火车不能继续运行,车上的乘客也会因此受到影响。而NBRE的升级方式相当于把零部件提前放在火车上,用一节车厢专门储存零部件,当火车发生故障之时,车上的工人就可以及时换上新的部件。在此过程中,火车仍可正常运作。

\textbf{奖励发放}

发布主网后自动发放所有相关人员的奖励。

