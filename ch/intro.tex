% !TEX root = main.tex

\section{星云简介}

星云链是开源公链,主创团队成立于2017年6月,星云主网于2018年3月30日上线。

星云是自治元网络(Autonomous Metanet),标志着协作的未来已来。

星云链使用超映射结构元数据解决复杂数据和交互的问题;使用全新的共识机制和升级能力解决复杂协作问题,提供持久的正向激励,无需硬分叉即可升级。

\textbf{关键字: 元数据\ 元网络\ 自治\ 自进化}

\subsection{背景}
2008年10⽉31⽇,中本聪(Satoshi Nakamoto)提出了⽐特币的设计⽩皮书\footnote{https://zh.wikipedia.org/wiki/\%E6\%AF\%94\%E7\%89\%B9\%E5\%B8\%81},从此我们迎来了一个有区块链的世界。

区块链技术本质上是⼀种去中心化、⾮信任、基于博弈的⾃治体系,其真正的魅力是在去中⼼化思想下,基于共识机制的开放协作模式\footnote{https://nebulas.io/docs/NebulasWhitepaperZh.pdf}。
当前,社区⽤户对于区块链未来有⾮常大的期望,但区块链还远没有达到最终演化形态。区块链创新迎来空前的黄金窗口,与此同时也面临诸多挑战:

\begin{itemize}
\item \textbf{信息交互复杂度与日俱增:}

信息交互包括链内和链外,如“跨链”的数据和资产交互问题等。区块链正在从单⼀、简单的功能向复杂、多样发展。比特币是一种去中心化的电子加密货币,区块链可以记录交易信息;以以太坊为代表的第二代区块链提出了具有图灵完备性的智能合约,区块链变得可编程。随着区块链技术不断发展,信息交互复杂度与日俱增。当前不同的区块链系统之间相互隔离,形成“数据孤岛”,传统的链式结构无法满足所有应用场景的需求。

\item \textbf{硬分叉困境,系统升级举步维艰:}

为了应对信息交互复杂度提升,系统升级变得非常重要。而现有区块链的版本迭代往往引发“硬分叉”或“软分叉”,各类规则一旦早期确定,后期很难进行改变。协议进化缓慢,发展步伐受阻。如果选择“硬分叉”,又相当于割裂和分化社区,以太坊的“硬分叉”就导致ETH和ETC“双重资产”和社区分裂的“副作用”。

\item \textbf{用户角色越来越多,共识机制遭遇挑战:}

从早期比特币矿工、持币者,到开发者、应用使用者等,越来越多的人接触到区块链,利益分配受到挑战。工作量证明(Proof of Wook,PoW)仅专注挖矿激励,无法满足不同角色用户的利益需要;权益证明(Proof of Stake)及其变体摇摆在去中心化和中心化之间。去中心化的以太坊饱受诟病的升级缓慢受制于其共识机制,需要广泛认同才可执行升级,而整个生态中各方意见难以统一。另一方面,EOS干脆通过核心仲裁法庭对黑客进行判决,直接注销账户,无法反驳,又被认为是中心化的管理模式,部分利益集团的寡头共识。
\end{itemize}

目前不存在完善的解决方案可以解决上述问题。我们意识到,新技术的诞生众望所归。



\subsection{星云是自治元网络}
自治元网络将是星云可见未来的形态。要介绍自治元网络,先从元数据说起。
\subsubsection{星云指数是特殊的元数据}
\textbf{元数据}(Metadata)是用以描述数据的数据。该名词起源于1969年。“Meta”即知识论中“关于”的意思。元数据包含三类,分别是记叙性元数据、结构性元数据和管理性元数据\footnote{https://zh.wikipedia.org/zh/\%E5\%85\%83\%E6\%95\%B0\%E6\%8D\%AE}。例如:
\begin{itemize}
\item 查看网页源代码时可以看到头部内容包含用以描述网页的标题、内容、关键词等的Meta信息,这是记叙性元数据;
\item Kindle里每本书的页码是结构性元数据;
\item 图书馆的书目索引系统便于读者查找图书,是管理性元数据。
\end{itemize}
区块链上也有元数据,例如:
\begin{itemize}
\item 比特币网络上的数据记录了交易地址、金额、时间、结果等交易信息,是一种记叙性元数据。
\item DApp的智能合约记录了许多描述信息和结构信息,也可以看作是元数据。DApp的iOS客户端、链下数据如图片、视频等,不是元数据。
\end{itemize}

\textbf{元数据的结构}可以是层级的;也可以是一维线性的,数据本身没有关联;还可以是二维的,比如与多个坐标相关的元数据。



\textbf{星云指数}(Nebulas Rank,NR)是一种满足真实性、公平性和多样性的区块链价值衡量标准,用以衡量账户地址对于区块链这一经济系统的贡献度\footnote{https://nebulas.io/docs/NebulasYellowpaperZh.pdf}。

星云指数对记录了数据描述信息、结构信息甚至管理信息的元数据进行再管理,试图在元数据与元数据之间建立关系。

星云指数也是用以描述数据的数据,故也是元数据。但星云指数超出了常见的一维或二维的数据结构,用单一平面无法描述,相当于地理学里新加图层的概念,是一种具有\textbf{超映射}\footnote{https://zh.wikipedia.org/zh/\%E5\%85\%83\%E6\%95\%B0\%E6\%8D\%AE\#\%E8\%B6\%85\%E6\%98\%A0\%E5\%B0\%84}(Hypermapping)结构的元数据。

\subsubsection{元网络是包含了超映射结构元数据的网络}
\textbf{元网络}(Metanet)特指包含了像星云指数这样的超映射结构元数据的网络。

除了反映数据对整个经济体的贡献度的核心星云指数,还有适应具体场景的扩展星云指数等具备超映射结构的元数据。

通过这些元数据,在元网络中,数据与数据的关系将更加标准化,丰富且复杂的数据组织关系将得到更好的描绘,跨越多层级进行数据传递、检索、比较等行为变得可行,“数据孤岛”被打破,现有数据价值得到全面衡量,甚至发现新的价值。

从链状结构向网状结构发展将是趋势,元网络有望应对日益丰富的数据类型和日益增长的需求。

\subsubsection{自治元网络}
\textbf{自治}(Autonomy)指自己治理自己\footnote{https://zh.wikipedia.org/wiki/\%E8\%87\%AA\%E6\%B2\%BB}。自治可能指个人行为,但通常指较大规模的行为,即以一个行业、行业团体、宗教、政府等为单位的行为。区块链的本质是自治体系。

\begin{enumerate}
	\item \textbf{星云激励是自治的基石}
	
	核心星云指数可以综合多种参数通过算法综合判断区块链上某一个账户地址对整个经济系统的贡献度。基于此,在元网络中,不仅仅是矿工和持币用户,开发者、活跃的使用者等对于整个生态的贡献度都可以得到相对公平的量化,互相之间还可以进行比较,有希望根据贡献度来激励该生态系统中的每个人。而且这种激励是基于算法的原生激励,不存在个别寡头统治整个网络的可能性。
	关于激励有2点基本认知:
	\begin{itemize}
		\item 正向激励是保证人人公平获益的基础。激励方向错误就会导致劣币驱逐良币。
		\item 激励需要持续,短视的激励方案同样会造成不可挽回的错误刺激。
	\end{itemize}
	\textbf{星云激励}(Nebulas Incentive)基于上述2个基本认知构建。星云会通过\textbf{开发者激励协议}(Developer Incentive Protocol,DIP)来提供原生开发者激励,鼓励开发者开发更好的应用,为社区做贡献;并通过\textbf{贡献度证明}(Proof of Devotion,PoD)共识机制来激励整个社区。
	
	持久的正向激励是社区组织的核心,自治的基石。
	
	\item \textbf{自进化是自治的未来}
	
	星云链另一个技术特点是无需硬分叉,可以自由演绎升级。具有|textbf{星云原力}(Nebulas Force)之后,就不会像早期区块链公链那样被共识机制或者不成熟的技术和策略束缚。
	
	基于元数据和以此为基础的更有效的激励和治理机制,结合星云原力,该元网络可以量化验证区块链系统的升级方案,确定系统升级方向,不断迭代,实现自我进化。
	
	\item \textbf{广义和狭义的自治元网络}
	
	\textbf{广义上说,自治元网络(Autonomous Metanet)是具有自组织\footnote{https://zh.wikipedia.org/wiki/\%E8\%87\%AA\%E6\%88\%91\%E7\%B5\%84\%E7\%B9\%94}特点的元网络。}在没有外部引导和管理下,可以自行增加复杂性,自行完成创生、组织、发展的全过程。
	
	自治元网络的构建目标是:
	\begin{itemize}
		\item 处理复杂数据和交互;
	    \item 处理复杂协作关系。
	\end{itemize}
	\textbf{狭义上说,当前来看,自治元网络是}一种基于区块链技术构建的,面向复杂数据和交互的开放协作体系。即,\textbf{实现了星云非技术白皮书所提核心技术的星云链}(预期为2020年)。
	
	自治元网络的升级换代将随着数据和协作关系复杂度提升而提升,不断迭代是其特有能力,正因如此,当前无法预测未来的自治元网络可以发展到何种程度。
\end{enumerate}


