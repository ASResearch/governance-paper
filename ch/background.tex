\section{背景}

星云治理的目的是实现星云愿景:让每个人从去中心化协作中公平获益。在谈论具体的星云治理措施之前,需要先了解当前新场景下出现的协作困境,以及星云的治理方式想要解决的问题。

人类是具有社会性的,我们对协作(Collaboration)并不陌生。即便是孤岛上的鲁滨逊,也和星期五磨合出了一套相处模式。协作方式本身没有绝对高低优劣之分,在不同场景下,应选择适合该场景的协作方式,或多种协作方式的组合。随着科技的发展,协作场景已经从人与人面对面协作,升级到全球化的跨越多地区、并有多组织参与的协作。协作的目标结果也更为多变,从实物发展到虚拟成果;协作的时间跨度也变得更长、更灵活。

星云并不力求颠覆其它协作方式,也不排斥多种协作方式混用。寻找一种更合适的协作方式,补足当前其它协作方式在新场景下的不足。新的应用场景指的是自比特币诞生以来,信息交互复杂度与日俱增、用户角色越来越多的新型社区,以链上资产为主要治理对象,链上交互为基本协作方式。

去中心化的电子加密货币比特币的诞生让区块链可以记录交易信息。以以太坊为代表的第二代区块链提出了具有图灵完备性的智能合约,区块链变得可编程。随着区块链技术不断发展,信息交互从单⼀、简单的功能向复杂、多样发展。逐步出现了,链内、链外、“跨链”等不同场景的数据和资产交互问题。当前不同的区块链系统之间相互隔离,形成“数据孤岛”。

在新场景下,用户角色也越来越多。早期比特币社区只有矿工和持币者,有了以太坊之后出现了开发者、应用使用者等,越来越多的人接触到区块链,利益分配受到挑战。例如比特币使用的共识机制工作量证明(Proof of Wook,PoW)仅专注挖矿激励,无法满足不同角色用户的利益需要。

区块链技术本质上是⼀种去中心化、⾮信任、基于博弈的⾃治体系,其真正的魅力是在去中⼼化思想下,基于共识机制的开放协作模式\footnote{https://nebulas.io/docs/NebulasWhitepaperZh.pdf}。传统去中心化协作项目如大量开源社区经费来源往往依靠募捐,升级进化依靠兴趣,会出现社区发展目标不明确、公地悲剧、效率低下、生态进化缓慢的问题。区块链技术有机会解决去中心化协作方式的困境。但当前的共识机制已不能满足生态发展的需要。以太坊的共识机制权益证明(Proof of Stake,PoS)及其各种变体(如DPoS)摇摆在去中心化和中心化之间。去中心化的以太坊饱受诟病的升级缓慢受制于其共识机制,升级提案需要得到社区广泛认同后才可执行,而整个生态中各方意见难以统一,激励机制也没有覆盖整个生态不同角色,导致升级提案参与度低,迟迟难以升级,生态发展受阻。

另一方面,有的区块链项目干脆反其道而行之,使用中心化的人治方式进行治理,如直接通过核心仲裁法庭直接对黑客进行判决。这种极端方式因其正当性和公平性难以得到保障,引起社区成员反抗。2019年1月11日,EOS Authority网站上发起关于是否应该废除ECAF(EOS核心仲裁法庭)的投票,支持废除的比例高达98\%\footnote{https://eosauthority.com/polls_details?proposal=decaf_20190111\&lnc=en}。

目前不存在完善的解决方案可以解决上述问题。我们意识到,新技术的诞生众望所归。

