\section{背景}

人类是具有社会性的,我们对协作(Collaboration)并不陌生。即便是孤岛上的鲁滨逊,也和星期五磨合出了一套相处模式,。协作方式本身没有绝对高低优劣之分,在不同场景下,应选择适合该场景的协作方式,或多种协作方式的组合。随着科技的发展,协作场景已经从人与人面对面协作,升级到全球化的跨越多地区、并有多组织参与的协作。协作的目标结果也更为多变,从实物发展到虚拟成果;协作的时间跨度也变得更长、更灵活。星云设计的带智能资产的去中心化协作并不力求颠覆其它协作方式,也不排斥多种协作方式混用。星云致力于在信息交互复杂度与日俱增、用户角色越来越多的全新应用场景下,寻找一种更合适的协作方式,补足当前其它协作方式在新场景下的不足,以便实现“让每个人从去中心化协作中公平获益”的愿景。

当前需要应对的新场景具有如下几个特点:

\begin{itemize}
\item \textbf{信息交互复杂度与日俱增}

以区块链为例。比特币是一种去中心化的电子加密货币,比特币的诞生让区块链可以记录交易信息。以以太坊为代表的第二代区块链提出了具有图灵完备性的智能合约,区块链变得可编程。随着区块链技术不断发展,信息交互从单⼀、简单的功能向复杂、多样发展。逐步出现了,链内、链外、“跨链”等不同场景的数据和资产交互问题。早期比特币时代并不存在此类情况。当前不同的区块链系统之间相互隔离,形成“数据孤岛”,传统的链式结构已无法满足所有应用场景的需求。

\item \textbf{硬分叉困境,系统升级举步维艰:}

为了应对信息交互复杂度提升,系统升级变得非常重要。而现有区块链的版本迭代往往引发“硬分叉”或“软分叉”,各类规则一旦早期确定,后期很难进行改变。协议进化缓慢,发展步伐受阻。如果选择“硬分叉”,又相当于割裂和分化社区,以太坊的“硬分叉”就导致ETH和ETC“双重资产”和社区分裂的“副作用”。

\item \textbf{用户角色越来越多,共识机制遭遇挑战}

从早期比特币矿工、持币者,到开发者、应用使用者等,越来越多的人接触到区块链,利益分配受到挑战。工作量证明(Proof of Wook,PoW)仅专注挖矿激励,无法满足不同角色用户的利益需要;权益证明(Proof of Stake)及其变体摇摆在去中心化和中心化之间。去中心化的以太坊饱受诟病的升级缓慢受制于其共识机制,需要广泛认同才可执行升级,而整个生态中各方意见难以统一。另一方面,有的区块链项目干脆使用中心化的管理模式,通过核心仲裁法庭直接对黑客进行判决。
\end{itemize}




\subsection{中心化协作的弊端}

中心化协作弊端:对上级负责,苹果等,依赖万能的一个人。这部分还没写完。

\subsection{去中心化协作的弊端}

原先去中心化协作的弊端:开源项目世纪难题,公地无人管理的问题。

智能资产引入去中心化协作之后的好处:不用当志愿者,自由意志得到发挥。

这部分还在写。

2008年10⽉31⽇,中本聪(Satoshi Nakamoto)提出了⽐特币的设计⽩皮书\footnote{https://zh.wikipedia.org/wiki/\%E6\%AF\%94\%E7\%89\%B9\%E5\%B8\%81},从此我们迎来了一个有区块链的世界。

区块链技术本质上是⼀种去中心化、⾮信任、基于博弈的⾃治体系,其真正的魅力是在去中⼼化思想下,基于共识机制的开放协作模式\footnote{https://nebulas.io/docs/NebulasWhitepaperZh.pdf}。
当前,社区⽤户对于区块链未来有⾮常大的期望,但区块链还远没有达到最终演化形态。区块链创新迎来空前的黄金窗口,与此同时也面临诸多挑战:


目前不存在完善的解决方案可以解决上述问题。我们意识到,新技术的诞生众望所归。

