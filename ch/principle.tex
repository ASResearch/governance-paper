\section{星云治理概述}
星云致力于通过技术手段,减小人为治理通常存在的摩擦和治理成本,用科技改变协作关系,促进社区良性发展,实现星云愿景,让每个人从去中心化协作中公平获益。

\subsection{星云愿景}
星云致力于让每个人从去中心化协作中公平获益\footnote{https://nebulas.io/cn/vision.html}。

\subsection{基本主张}
一切复杂系统的演绎从制定符合“常识”的基本规律开始。区块链资产的基本构成单位是“地址”,也是星云社区治理的基本单位。在此,我们正式提出关于星云链“地址”三点基本权利主张:
\begin{enumerate}
	\item 拥有和操作星云链上资产的权利;
	\item 发起提案的权利;
	\item 参与提案投票的权利。
\end{enumerate}
星云坚信每个地址在系统中拥有的以上基本权利,在任何条件下不受侵犯。而每个地址所对应的私钥是该地址控制权的唯一凭证,通过它,每一位星云社区成员都享有使用星云主网、参与社区决策、参与生态建设的权利。没有绝对中心化组织或个人,任何社区成员都可以提出提案,通过社区链上投票进行治理。星云未来发展方向的决定权在每一位星云社区成员手中。

\subsection{治理范围}
区块链是承载协作关系的网络,亦是承载激励协作关系“资产”的网络,在没有绝对中心化权力的网络里,相应的公共资产理应属于全体社区成员共同管理。

星云治理所针对的公共资产主要包括:
\begin{enumerate}
	\item 公共开源代码知识产权(如主网升级等影响到星云链上公共利益的相关代码);
	\item 白皮书公示的预留公共资金池。
\end{enumerate}

