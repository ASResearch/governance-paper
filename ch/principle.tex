\section{星云治理概述}

星云治理以链上治理为核心,以链上资产为主要治理对象,链上交互为基本协作方式,解决上述新场景下的治理困境,设计更好的去中心化协作方式,以实现星云愿景。

\subsection{技术基础}

开源公链星云链在2017年设计之初便围绕实现星云愿景而展开,为星云治理的实施打下技术基础。

星云是自治元网络,标志着协作的未来已来。星云链使用超映射结构元数据解决复杂数据和交互的问题,拥有衡量链上数据价值的核心能力(星云指数,Nebulas Rank,NR);使用全新的共识机制和升级能力解决复杂协作问题,提供持久的正向激励(星云激励,Nebulas Incentive,NI),且无需硬分叉即可升级(星云原力,Nebulas Force,NF)。星云致力于通过技术手段,减小人为治理通常存在的摩擦和治理成本,用科技改变协作关系,促进社区良性发展。详细星云技术特点请参考《星云技术白皮书》~\cite{TechWhitepaper}。

\subsection{基本权利主张}
\label{rights}

一切复杂系统的演绎从制定符合“常识”的基本规律开始。区块链资产的基本构成单位是“地址”(Address),也是星云治理的基本单位。在此,我们正式提出关于星云链“地址”三点基本权利主张:

\begin{enumerate}
	\item 拥有和操作星云链上资产的权利;
	\item 发起提案的权利;
	\item 参与提案投票的权利。
\end{enumerate}

星云坚信每个地址在系统中拥有以上基本权利,在任何条件下不受侵犯。每个地址所对应的私钥是该地址控制权的唯一凭证,通过它,每一位星云社区成员都享有使用星云主网、参与社区决策、共建星云生态的权利。

星云治理基于此三点基本权利主张进行。任何社区成员都可以提出提案,通过社区链上投票进行治理,星云未来发展方向的决定权在每一位星云社区成员手中。

\subsection{治理范围}

星云治理所针对的公共资产主要包括:

\begin{enumerate}
	\item 公共开源代码知识产权(如主网升级等影响到星云链上公共利益的相关代码);
	\item 《星云非技术白皮书》公示的预留公共资金池。
\end{enumerate}

区块链是承载协作关系的网络,亦是承载激励协作关系“资产”的网络,在没有绝对中心化权力的网络里,相应的公共资产理应属于全体社区成员共同管理。

但同时,星云治理的范围仅限星云公共资产,并非所有星云链上资产。星云治理为星云社区提供基本治理手段,星云社区中的组织机构(如DApp项目方、交易所等)可以使用星云治理工具(如NAT链上投票)促进自己项目的生态良性发展,但星云三会并不会“出任”法官角色去进行“人治”。此外,星云社区成员的链下行为应同时遵守当地法律法规。在第一章中已经描述,不同场景应采用相匹配的治理方式,星云治理并不会违背设计初衷,盲目扩张其治理范围。

\subsection{治理特点}

星云治理有如下三个特点:

\begin{enumerate}
	\item 

	\textbf{规则全息,公开透明。} 

	所有人在统一的规则之下存续和发展,与此同时,迭代规则的规则也被统一的规则定义。


	\item 

	\textbf{良性经济的去中心化协作。}

	\begin{enumerate}
		\item 

		\textbf{社区协作的过程去中心化:} 星云治理流程以链上治理为核心,辅助社区成员监督,没有“人治”。
	
		\item 

		\textbf{公共资产的治理去中心化:} 作为带有资产属性的去中心化社区:

		\begin{itemize}
			\item 通过星云三会保障流程正当性和权力相互制约,没有组织或个人具有至高权力,没有组织或个人可以直接动用公共资产。
			\item 通过独创的贡献度证明共识机制(Proof of Devotion,PoD)对资产分发和安全性提供技术保障。
		\end{itemize}

	\end{enumerate}

	\item 

	\textbf{激励完备,社区参与度高。}
	
	持久的正向激励是社区组织的核心,自治的基石。

	星云独创的核心星云指数(Core Nebulas Rank)可以综合多种参数通过算法综合判断区块链上某一个账户地址对整个经济系统的贡献度~\cite{yellowpaper}。基于此,不仅仅是矿工和持币用户,开发者、活跃的使用者等不同角色对于整个生态的贡献度都可以得到相对公平的量化,互相之间还可以进行比较,使得星云可以根据贡献度来激励该生态系统中的每个人。

	并且,星云“资产化”星云指数,通过提高经济体贡献度、积极参与链上治理(如链上投票)等方式,星云用户可以通过为社区做贡献而获得相应的NAT奖励,得到正向激励。这种激励是基于算法的原生激励,通过技术保障,而非个人指挥来执行,不存在个别寡头统治整个网络的可能性。

	关于激励有3点基本认知:

	\begin{itemize}
		\item 正向激励是保证人人公平获益的基础,激励方向错误就会导致劣币驱逐良币。
		\item 激励需要持续,短视的激励方案同样会造成不可挽回的错误刺激。
		\item 激励的规模是适当的,不应被恶意使用的。
	\end{itemize}

	星云在设计技术特点和星云经济体时都始终将激励视为重要组成部分。正向激励将有望让社区成员更公平地获益,大幅提高社区参与度。
	
	\item 

	\textbf{包容性强,协作效率高。}
	
	基于星云自治元网络无需硬分叉可自进化的特点,在星云社区,一个提案一旦通过星云社区公开链上投票,可即刻升级,快速迭代。遇到问题也可快速再次尝试,拥抱复杂性,容错性强,不会像早期区块链系统,如以太坊那样的公链被共识机制或者不成熟的技术和策略束缚。

	在技术效率高的同时,星云治理亦具有简单清晰的流程(参见\ref{governance}),从而提高协作效率。

\end{enumerate}
