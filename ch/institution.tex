\section{星云三会}

为了达成星云生态发展和资产治理目标,推进星云迈入“自治元网络”,星云创始团队将同社区一起组建“星云三会”。组建过程中将严格约定每个组织的权力正当性来源、组织方式、权力边界,且保证三会之间实现相互制约。星云三会分别为:
\begin{enumerate}
	\item 星云理事会:监督星云治理流程和星云社区公共资产使用的正当性,为星云生态发展提供规模优势;
	\item 星云基金会:管理所属基金会的资产,聚集资源,运用资本,为星云生态发展提供效率优势;
	\item 星云技术委员会:受星云理事会委托,负责星云项目制生产力组织及质量核查,为社区提供技术指引和支持。
\end{enumerate}
三个机构独立运行且互相制约。为了保证三个机构的独立性,维持三者之间的制衡关系,有两个基本原则:
\begin{enumerate}
	\item 人员权力制约:所有机构向星云社区所有人开放,但每名社区成员不能同时在两个以上机构中担任职务。
	\item 机构权力制约:任意机构不具有独立决策和使用星云公共资产的权力。
\end{enumerate}
如有必要引入新的原则,应始终保证三个机构独立运行且互相制约。

\subsection{星云理事会}
星云理事会监督星云治理流程和星云社区公共资产使用的正当性,为星云生态发展提供规模优势。

\subsubsection{人员组成}

第一届星云理事会理事将设置7席,其中,由星云基金会提名3席,通过社区链上公开投票选举产生4席。

星云基金会提名席位每两年至少减少一席。最晚6年后星云基金会不再提名席位。

\subsubsection{权力}
\begin{enumerate}
	\item 提请在星云生态中进行“二次投票”的权力。
	\item 委托星云技术委员会等组织机构或个人处理星云社区公共事务。。
\end{enumerate}

\subsubsection{义务}
星云理事会应确保治理流程和社区公共财产使用公开透明,包括但不限于:
\begin{enumerate}
	\item 定期通过季报等披露材料向社区公示资产使用及社区发展等情况。
	\item 如产生技术升级、项目申请驳回重新投票等处理情况,应及时公示。
	\item 所有人员选举和委任情况应及时公示。
\end{enumerate}

\subsubsection{任期}
星云理事会理事任期为2年,可连任1次。

社区成员对星云理事会有监督权。星云理事会理事任期满一年需要述职。社区根据述职情况,进行中期投票,未通过中期投票的理事将失去继续任职的资格。

在发生有理事未能通过中期投票的情况时,则由星云技术委员会组织并监督新一届星云理事会选举。并由述职通过的理事暂时代理星云理事会日常事务,直至新一届星云理事会理事选举完成。

\subsubsection{选举方法}
除星云基金会提名的星云理事会理事通过社区链上公开投票选举产生。只要拥有至少一个星云主网NAS账户地址的社区成员,都有选举与被选举权。

首期星云理事会选举方案由星云基金会代为提出和监督,并向社区征求意见。后续理事会选举方案可以由社区提起提案迭代。此类提案和星云生态中的其余提案一样,需经过社区链上公开投票通过后方可执行。


\subsubsection{收益}
\textbf{每一任期收入}

如两年任期期满,可获得:10,000 NAS

\textbf{收益发放}

每满半年分发一次,每一任期即2年内分4次分发。每次分发额度为:1,500 NAS、2,000 NAS、3,000 NAS、3,500 NAS。如中期述职投票没有通过,则后两期收入不予发放。

\textbf{财务要求}

为保证经济体利益的一致性,和星云理事会政策的延续性,星云理事会理事正式入职时需要抵押100,000 NAS,卸任半年后解锁返还。


\subsection{星云基金会}
星云创始团队于2017年6月组建,星云基金会随之成立,负责星云团队的运维及团队成员的期权,保障项目正常进行,实现《星云非技术白皮书》(2017)中的发展路线图。

在《星云非技术白皮书》中承诺的技术点全部实现之后,星云基金会将管理所属基金会的资产,聚集资源,运用资本,为星云生态发展提供效率优势。

\subsubsection{人员组成}
星云基金会常务理事不少于5席。其中,含星云基金会主席1席,秘书长1席。

\subsubsection{权力}
\begin{enumerate}
	\item 基金会内部席位的选举和被选举权。
	\item 参与基金会发展和投资等的决策。
\end{enumerate}
\textbf{义务}
\begin{enumerate}
	\item 管理所属星云基金会的资产。
	\item 按照星云发展需求组织生产,保障星云项目研发可以正常进行,按时完成《星云非技术白皮书》(2017)中的发展路线图。
	\item 每年一次对星云理事会述职,持续为星云生态服务。
\end{enumerate}



\subsubsection{任期}

星云基金会常务理事任期为1年,可连任。

\subsubsection{入选方法}
\textbf{星云基金会常务理事}

星云基金会采用按资入围制。享有期权规模达到一定额度的人员自动拥有成为星云基金会常务理事的资格。每位具有成为星云基金会常务理事资格的人员拥有放弃成为星云基金会常务理事的权利。当星云基金会常务理事不足5名时,则依照奖励额度排位候补。

\textbf{星云基金会主席}

星云基金会主席在星云基金会内部匿名推选而得。每位星云基金会常务理事具有星云基金会主席被选举权和选举权。

星云基金会主席支持率需要过半。如果无人支持率过半,则末位淘汰后重新投票。

\textbf{星云基金会秘书长}

星云基金会秘书长由星云基金会主席在现任星云基金会成员中委任。

\textbf{罢免}

星云基金会可通过内部决议罢免任意一名星云基金会常务理事,但结果必须向社区公示。

被罢免的星云基金会常务理事有权向社区公开述职,并有一次机会发起链上社区公开投票请求恢复职务。


\subsubsection{收益}
\textbf{总收入}

\begin{enumerate}
	\item 星云基金会薪资及相关期权奖励。
    \item 星云基金会生态投资等相关业务带来的收益。
\end{enumerate}

\textbf{财务要求}

为保证经济体利益的一致性,和星云基金会政策的延续性,星云基金会常务理事正式入职时需要抵押50,000 NAS,卸任半年后解锁返还。

\subsection{星云技术委员会}
星云技术委员会成立于2018年9月。成立至今,星云技术委员会秉持着开放、共享、透明的精神,致力于推动星云链技术研发逐步去中心化、社区化。

自星云理事会成立后,原由星云团队核心成员构成的星云技术委员会将完成历史使命,转型为社区化团队。星云技术委员会受星云理事会委托,负责星云项目制生产力组织及质量核查,为社区提供技术指引和支持。

\subsubsection{人员组成}
星云技术委员会人数不限。

在2019年星云理事会第一次选举完成之前,星云技术委员会暂由星云团队成员代理。自第一届星云理事会选举结束后,星云技术委员会将启动组建。

\subsubsection{权力}
\begin{enumerate}
	\item 向理事会发起提案重审的权力。
	\item 享有星云技术专家团队的荣誉。
\end{enumerate}

\textbf{义务}

\begin{enumerate}
	\item 社区化提案质量监督,出具相关测试报告和技术评级报告。
\end{enumerate}

\subsubsection{任期}

技术委员会委员任期1年,可连任。

\subsubsection{入选方法}
技术委员会采用自荐和社区推荐相结合的方法选择,对社区公开述职。具体将由星云理事会组织进行。

\subsubsection{收益}
\textbf{总收入}
\begin{itemize}
	\item 委托佣金(按月发放)
	\item 项目审核和监督的咨询费用
\end{itemize}

\textbf{财务要求}

为保证经济体利益的一致性,和星云技术委员会政策的延续性,星云技术委员会委员正式入职时需要抵押25,000 NAS。卸任3个月后解锁返还。