
\section{NAT发行算法}

NAT的发行是根据每个用户的星云指数、投票行为以及质押情况决定的,没有预发行。

\subsection{概述}
NAT的发行按照星云指数的计算周期进行(注意,投票周期和星云指数周期相同),也就是说,在每个星云指数的计算周期结束,根据该星云指数计算周期每个用户的星云指数情况及投票、质押行为进行发放。
基本来说,
对于周期$i$,系统中新增的NAT $\mathcal{T}_i$分为三个部分:根据地址的星云指数获得的部分$\mathcal{A}_i$、投票激励部分$\mathcal{V}_i$、质押激励部分$\mathcal{D}_i$。
另外,用户用于投票的NAT会被销毁一定的比例,假设对于周期$i$,系统中因投票减少的NAT为$\mathcal{M}_i$,则系统中总的NAT发行量为:
\begin{align}
\sum_{i=1}^{\infty} (\mathcal{A}_i + \mathcal{V}_i + \mathcal{D}_i - \mathcal{M}_i)
\end{align}

为方便说明,此处先给出本章涉及到的符号,并给出相应的说明,
\begin{itemize}
\item $\mathcal{C}_i$:系统在周期$i$的星云指数总和;
\item $c_{i,j}$:用户$j \in \mathcal{U}$在周期$i$的星云指数值;
\item $d_{i,j}$:用户$j \in \mathcal{U}$在周期$i$质押的NAS总量;
\item $v_{i,j}$:用户$j \in \mathcal{U}$在周期$i$投票的NAT总量。
\end{itemize}

\subsection{根据地址的星云指数获得的部分}
此部分与用户的星云指数相关,定义为:
\begin{align}
    f(x) = g(x)\lambda^i
\end{align}
\noindent 其中$x$为用户星云指数;$g(x)$为调整NAT总量与星云指数总量关系的比例函数,且满足$g(0) = 0$;$\lambda$为衰减系数,且$\lambda < 1$。
由于$\lambda < 1$,易知$\lim_{i\to \infty}f(x) = 0$。

可知,此部分在周期$i$的新增总量为:
\begin{align}
\mathcal{A}_i = \sum_{i=1}^{\infty}f(\mathcal{C}_i)。
\end{align}

\subsection{投票激励部分}
投票激励部分和用户的投票行为,以及星云指数相关,对于用户$j \in \mathcal{U}$,投票激励为:
\begin{align}
\mu f(x_{i-1,j}) \min\{\frac{v_{i,j}}{f(x_{i-1,j})},1\}
\end{align}
\noindent 其中$\mu$为投票激励系数,$\mu > 1$,表示对用户的投票行为给予额外的奖励,可以根据系统中流通的NAS的数量变化调整。

\subsection{质押部分}
质押部分获得的NAT应与部分NAS提升星云指数获得的部分存在相关性。根据星云指数的性质可知,给定NAS,其存在星云指数的上限$h(d_{i,j})$~\cite{ImproveNR},

则定义质押部分获得的NAT为:
\begin{align}
\mathcal{D}_i = \sum_{i=1}^{\infty}\alpha f(h(d_{i,j}))
\end{align}
\noindent 其中$\alpha$为质押激励系数。


\subsection{销毁部分}
\label{burn}
用户每次投票,都会有一部分被销毁,剩余的部分返还给用户,同时,NAT项目团队为了支付投票活动的必要开销,对每笔投票征收$\theta\%$的费用。因此,对每个用户而言,定义销毁部分为:
\begin{align}
(1-\theta\%) \times \beta^i \times v_{i,j}
\end{align}
\noindent 其中,$\beta$为销毁系数,且$\beta < 1$。因此,
\begin{align}
    \mathcal{M}_i = \sum_{i=1}^{\infty} (1-\theta\%) \times \beta^i \times v_{i,j} 。
\end{align}

\subsection{分析}

注意:
\begin{itemize}
\item 目前版本暂定赞同票与反对票没有区别,即返还比例相同。之后可根据票种设定并乘上不同的返回参数$\mu_1$;
\item 若考虑到投票完成后系统的总星云指数变化,则可再乘上一个系数$\mu_2$,用于反应该周期系统的繁荣度。
\end{itemize}


\begin{property}
本算法能满足NAT总量的收敛性,即NAT总量在任何时候都不会超过一个上限。
\end{property}
\begin{proof}
	根据《星云技术白皮书》的设定,NAS的固定总量为$10^9$,平均每周大约增发(在固定总量的基础上)$0.2\%$,故在第$n$个投票周期市面上现存NAS总量不会超过$10^9(1+0.002n)$。
	
	接下来我们证明所有地址一个周期内的资产中值(见《星云指数黄皮书》中的定义)总和不会超过市面上现存NAS总量。这是因为,对于任意一笔数量为$y$的NAS资产,他只能最多在一个地址内存在该周期一半以上的时间(三天半),故最多给全网节点的总资产中值提供$y$的贡献。
	
	同样根据星云指数黄皮书的设定,任何一个地址的星云指数值不会超过该地址的资产中值(指同样一个周期内,注意星云指数和NAT的计算都是以周为单位,具有同步性),这是因为黄皮书星云指数计算公式$\Omega(\cdot)\Psi(\cdot)$中,以资产中值为输入的Wilbur函数$\Omega(\cdot)$满足$\Omega(x)\leq x$,且出入度函数$\Psi(\cdot)$值域不超过1。
	
	结合上述结论,可得在第$n$个周期,所有地址星云指数总和不超过$10^9(1+0.002n)$,从而根据地址星云指数获得的NAT不超过$g(10^9(1+0.002n))\lambda^n$。
	
	又因为投票激励部分的NAT不超过增发部分乘以$\mu$,故即使加上返还部分带来的增量,周期$n$内激励部分NAT的总增量不超过$\mu g(10^9(1+0.002n))\lambda^n$。另外,质押部分带来的增量不超过NAS总量$g(10^9(1+0.002n))\lambda^n$。
	
	最后,欲证NAT总量的收敛性,因为根据地址的星云指数获得的部分、质押部分和激励部分均随时间指数级衰减,故只需证明级数:
	\begin{align}
	\sum_{n=1}^{\infty} \mu g(10^9(1+0.002n))\lambda^n
	\end{align}
	收敛。由于$g(\cdot)$为线性函数,故
	\begin{align}
	\lim_{n\rightarrow \infty} \frac{\mu g(10^9(1+0.002(n+1)))\lambda^{n+1}}{\mu g(10^9(1+0.002n))\lambda^n} = \lambda <1
	\end{align}
	由比式判别法可得该级数收敛,证毕。
\end{proof}
同时,上述投票算法具有下列良好性质。
\begin{enumerate}
	\item \textbf{抗滚雪球效应:} 如若简单的按固定比例返还NAT,则一个用户可以每次投出所有的NAT并享受大于1比例的返还(如1.1),则其总NAT将按$1.1^n$指数级上升,增长过于庞大。
	\item \textbf{抗收买性:} 若一个低星云指数的用户以购买的方式获取大量NAT并用于投票,由于对低星云指数用户我们设定的对应$x_{i-1}^j$较低,返还的NAT很少,大部分都被烧毁,导致该用户剩余NAT很少作为惩罚。
	\item \textbf{抗通货膨胀:} 由于系统增发NAT比例与当前市场NAT总量有关,可有效控制NAT的贬值。
	\item \textbf{头部效应:} 早期拥有高星云指数的用户能拥有更高NAT总量。
\end{enumerate}