
\section{NAT发行算法}

NAT的发行是根据每个用户的NR、投票行为以及质押情况决定的,没有预发行。

\subsection{概述}
NAT的发行按照NR的计算周期进行(注意,投票周期和NR周期相同),也就是说,在每个NR的计算周期结束,根据该NR计算周期每个用户的NR情况及投票、质押行为进行空投。
基本来说,
对于周期$i$,系统中新增的NAT $\mathcal{T}_i$分为三个部分,空投部分$\mathcal{A}_i$,投票激励部分$\mathcal{V}_i$,质押激励部分$\mathcal{D}_i$。
另外,用户的投票会销毁一定的比例,假设对于周期$i$,系统中因投票减少的NAT为$\mathcal{M}_i$,则系统中总的NAT发行量为
\[
\sum_{i=1}^{\infty} (\mathcal{A}_i + \mathcal{V}_i + \mathcal{D}_i - \mathcal{M}_i)
\]

为方便说明,此处先给出本章涉及到的符号,并给出相应的说明,
\begin{itemize}
\item $\mathcal{C}_i$:系统在周期$i$的NR总和;
\item $c_{i,j}$:用户$j \in \mathcal{U}$在周期$i$的NR值;
\item $d_{i,j}$:用户$j \in \mathcal{U}$在周期$i$质押的NAS总量;
\item $v_{i,j}$:用户$j \in \mathcal{U}$在周期$i$投票的NAT总量。
\end{itemize}

\subsection{空投部分}
空投部分与用户的NR相关,即
\[
    f(x) = g(x)\lambda^i
\]
\noindent 其中$x$为NR,可以为某个用户的NR;$g(x)$为调整NAT总量与NR总量关系的比例函数,且满足$g(0) = 0$;$\lambda$为空投衰减系数,且$\lambda < 1$。
由于$\lambda < 1$,易知$\lim_{i\to \infty}f(x) = 0$。

可知,空投部分在周期$i$的新增总量为
\[
\mathcal{A}_i = \sum_{i=1}^{\infty}f(\mathcal{C}_i)。
\]

\subsection{投票激励部分}
投票激励部分和用户的投票行为,以及NR相关,对于用户$j \in \mathcal{U}$,投票激励为
\[
\mu f(x_{i,j}) \min\{\frac{v_{i,j}}{f(x_{i-1,j})},1\}
\]
\noindent 其中$\mu$为投票激励系数,$\mu > 1$,表示对用户的投票行为给予额外的奖励,可以根据系统中流通的NAS的数量变化调整。

\subsection{质押部分}
质押部分获得的NAT应少于使用该部分NAS提升NR获得的空投部分。根据NR的性质可知,给定NAS,其存在NR的上限$h(d_{i,j})$\footnote{\url{https://community.nebulas.io/d/175-nas-nr}},
则质押部分获得的NAT为
\[
\mathcal{D}_i = \sum_{i=1}^{\infty}\alpha f(h(d_{i,j}))
\]
\noindent 其中$\alpha$为质押激励系数。


\subsection{销毁部分}
用户每次投票,都会有一部分被销毁,剩余的部分返还给用户,同时,星云理事会为了支付投票活动的必要开销,对每笔投票征收1\%的费用。因此,对每个用户而言,销毁部分为
\[
0.99 * \beta^i * v_{i,j}
\]
\noindent 其中,$\beta$为销毁系数,且$\beta < 1$。因此,
\[
    \mathcal{M}_i = \sum_{i=1}^{\infty} 0.99 * \beta^i * v_{i,j} 。
\]

\subsection{分析}

注意:
\begin{itemize}
\item 目前版本暂定赞同票与反对票没有区别,即返还比例不同。之后可根据票种设定并乘上不同的返回参数$\mu_1$
\item 若考虑到投票完成后系统的总NR变化,则可再乘上一个系数$\mu_2$,用于反应该周期系统的繁荣度。
\end{itemize}


\begin{property}
本算法能满足NAT总量的收敛性,即NAT总量在任何时候都不会超过一个上限。
\end{property}
\begin{proof}
	根据星云链技术白皮书的设定\footnote{https://nebulas.io/docs/NebulasTechnicalWhitepaperZh.pdf},NAS的固定总量为$10^9$,平均每周大约增发(在固定总量的基础上)$0.2\%$,故在第$n$个投票周期市面上现存NAS总量不会超过$10^9(1+0.002n)$。
	
	接下来我们证明所有地址一个周期内的资产中值(见星云指数黄皮书中的定义\footnote{https://nebulas.io/docs/NebulasYellowpaperZh.pdf})总和不会超过市面上现存NAS总量。这是因为,对于任意一笔数量为$y$的NAS资产,他只能最多在一个地址内存在该周期一半以上的时间(三天半),故最多给全网节点的总资产中值提供$y$的贡献。
	
	同样根据星云指数黄皮书的设定,任何一个地址的NR值不会超过该地址的资产中值(指同样一个周期内,注意NR,NAT的计算都是以周为单位,具有同步性),这是因为黄皮书NR计算公式$\Omega(\cdot)\Psi(\cdot)$中,以资产中值为输入的Wilbur函数$\Omega(\cdot)$满足$\Omega(x)\leq x$,且出入度函数$\Psi(\cdot)$值域不超过1。
	
	结合上述结论,可得在第$n$个周期,所有地址NR总和不超过$10^9(1+0.002n)$,从而NAT空投部分不超过$g(10^9(1+0.002n))\lambda^n$。
	
	又因为投票激励部分的NAT不超过增发部分乘以$\mu$,故即使加上返还部分带来的增量,周期$n$内NAT的总增量超过$\mu g(10^9(1+0.002n))\lambda^n$。另外,质押部分带来的增量不超过NAS总量$g(10^9(1+0.002n))\lambda^n$。
	
	最后,欲证NAT总量的收敛性,因为空投部分,质押部分和激励部分均随时间指数级衰减,故只需证明级数
	$$\sum_{n=1}^{\infty} \mu g(10^9(1+0.002n))\lambda^n$$
	收敛,由于$g(\cdot)$为线性函数,故
	$$ \lim_{n\rightarrow \infty} \frac{\mu g(10^9(1+0.002(n+1)))\lambda^{n+1}}{\mu g(10^9(1+0.002n))\lambda^n} = \lambda <1$$
	由比试判别法可得该级数收敛,证毕。
\end{proof}
同时,上述投票算法具有下列良好性质。
\begin{enumerate}
	\item 抗滚雪球效应:如若简单的按固定比例返还NAT,则一个用户可以每次投出所有的NAT并享受大于1比例的返还(如1.1),则其总NAT将按$1.1^n$指数级上升,增长过于庞大。
	\item 抗收买性:若一个低NR的用户以购买的方式获取大量NAT并用于投票,由于对低NR用户我们设定的对应$x_{i-1}^j$较低,返还的NAT很少,大部分都被烧毁,导致该用户剩余NAT很少作为惩罚。
	\item 抗通货膨胀:由于系统增发NAT比例与当前市场NAT总量有关,可有效控制NAT的贬值。
	\item 头部效应:早期拥有高NR的用户能拥有更高NAT总量
\end{enumerate}