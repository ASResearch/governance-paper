\section{监督}
\subsection{社区监督}
社区所有人可以参与监督执行,并可以向技术委员会提出调整建议。包含但不限于如下情况:
\begin{enumerate}
\item 如发现此前方案不完善,并发现了更好的优化方案。
\item 遭遇意外突发情况如黑客袭击、人员离世等,导致原定计划无法执行,项目被迫中断。
\item 在进入执行阶段后被调查核实出,此前公开提案和投票中有违反社区治理基本准则、各国法律法规、或舞弊等违规情况。
\end{enumerate}

\subsection{星云技术委员会监督}
由技术委员会听取执行人、提案人、提出调整建议者、其他相关人员等多方建议,对后续执行方向作出判断。
\subsection{星云理事会监督}
技术委员会驳回提案需要向理事会申请。理事会可发起一次重新投票。每个项目最多重新投票1次。
星云理事会发起二次投票采用多重签名的方案:7签4~7签7,则资金可用。和投票机制一样,参考NAS当前流动性和总额进行加权,涉及使用的金额越大,需要的签名就越多。
